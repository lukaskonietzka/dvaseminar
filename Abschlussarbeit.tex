%%% ---------------------------------------------------------------------------------
%%% Vorlage Abschlussarbeit (LaTeX)
%%%
%%% V1   03/2017, Stefan Etschberger (HSA)
%%% V1.1 04/2021, rnw-hack für biblatex-run
%%% V2   05/2021, Titelblatt und Erweiterungen: Stefan Jansen (HSA)
%%% V2.1 05/2021, Trennung von R-Support und einfachem LaTeX: Phillip Heidegger (HSA)
%%% V2.2 01/2024, Anpassung an THA-Layout
%%% V3   01/2024, I18n
%%% ---------------------------------------------------------------------------------
\documentclass[
	12pt,
	a4paper %
	,
	twoside % Fuer Veröffentlichung
	,
	titlepage,
	DIV=13,
	headinclude,
	footinclude=false %
	,
	cleardoublepage=empty %
	,
	parskip=half,
]{scrreprt}

\usepackage[utf8]{inputenc}
\usepackage[T1]{fontenc}
\usepackage[hidelinks]{hyperref}
\usepackage{xcolor}

\usepackage[
	authorName={Lukas Konietzka},
	authorEnrolmentNo={2122553},
	authorStreet={Sebastian-Kneipp-Gasse 6A},
	authorZip={86152},
	authorCity={Augsburg},
	authorEMail={lukas.konietzka@tha.de},
	authorPhone={+49\,172-2728-376},
	authorSignaturePlace={Augsburg},
	studyProgram={},
	thesisType={Seminararbeit},
	thesisTitle={SQL Analyser-Erweiterungen},
	studyDegree={Bachelor of Science Informatik \\Im Rahmen des DVA-Seminars},
	faculty={{Fakultät für \\ Informatik}},
	%           ,topicAssignment={\today}
	submissionDate={10. Januar 2025}
	%           ,defenseDate={\today}
	%           ,nonDisclosure={false}
	,
	supervisor={Prof.~Matthias ~Kolonko, ~Ph.D. (ONPU)}
	%           ,supervisorDeputy={Prof.~Dr.~Mario Huana}
	%           ,language={en}
]{THA-Abschlussarbeit}

% Literaturdatenbank (.bib-Datei) aus Citavi o.ä.
\bibliography{Literatur_Abschlussarbeit}

\graphicspath{{Bilder/}}

\usepackage{caption}
\DeclareCaptionLabelFormat{something}{#2.#1.}
\captionsetup[lstlisting]{labelformat=something}

\begin{document}
	% Sprachauswahl zum Umschalten innerhalb des Textes.
	% Alternativen: \thesisLanguage, ngerman, english
	\selectlanguage{\thesisLanguage}
	\pagenumbering{roman}

	\THAtitlepage

	\let
	\cleardoublepage
	\relax

	%%% --------------------------------------------------
	%%% Kurzfassung und Inhaltsverzeichnis
	%%% --------------------------------------------------

	\begin{abstract}
		\section*{Kurzfassung}
		Die Sprache SQL (Structured Query Language) verfügt über diverse
		Erweiterungen für die unterschiedlichsten Einsätze. Eine der wichtigsten Erweiterungen
		sind die analytischen Fähigkeiten, auch bekannt als Analyse-Erweiterungen. Diese
		finden ihren Einsatz vermehrt in der Datenanalyse und stoßen deshalb bei Unternehmen
		auf immer größer werdendes Interesse. Dieser Artikel soll die Möglichkeiten
		und Sprachkonstrukte dieser Erweiterungen näher beleuchten. Hierfür werden zwei
		der vielen Analyse-Erweiterungen ausgewählt und detailliert betrachtet. Im
		ersten Schritt wird eine umfassende Literaturrecherche zum aktuellen Stand
		der Technik durchgeführt. Diese gibt Auskunft über die Funktion der hier fokussierten
		Erweiterungen. Im Mittelpunkt der Arbeit stehen die
		Beispielimplementierungen, die anhand der \textit{LibraryDB} veranschaulicht
		werden. Die genauere Betrachtung dieser konkreten Beispiele wird zeigen,
		dass die Analyse-Erweiterungen innerhalb von SQL zwar vielseitige
		Einsatzmöglichkeiten und mächtige Sprachkonstrukte bieten, jedoch auch mit einer
		erhöhten Komplexität einhergehen.
	\end{abstract}

	%%% --------------------------------------------------
	%%% Inhaltsverzeichnis
	%%% --------------------------------------------------

	\tableofcontents

	%%% --------------------------------------------------
	%%% Inhalt
	%%% --------------------------------------------------

	\setcounter{page}{1}
	\pagenumbering{arabic}

	\chapter{Einleitung}
\label{chap:einleitung} Der SQL-Standard wird bald 40 Jahre alt und hat bis heute
diverse Erweiterungen erfahren. Ein der wichtigsten sind die Analyse-Erweiterungen.
Einige dieser Analyse-Fähigkeiten haben es im Laufe der Jahre in den SQL-Standard
geschafft und sind ab diesem Zeitpunkt ein fester Bestandteil von SQL. Trotz
Ihrer doch langen Geschichte, stehen die Analyse-Erweiterungen eher im Schatten der
breiten Palette an SQL-Befehlen und konnten nicht sonderlich viel an Bekanntheit
gewinnen.
% --------------------------------------------------------------------------------------------

\section{Ziel der Arbeit}
\label{sec:ziel_der_arbeit} Diese Arbeit soll für Klarheit sorgen und das Geheimnis
dahinter offenbaren. Es werden einige dieser Analyse-Erweiterungen anhand
einfacher Beispiele aufzeigen, um eine eingehende Erklärung zu bieten. Hierzu
sollen die unterschiedlichen Sprachkonstrukte und Methoden demonstriert und die
Konzepte dahinter erläutert werden. Es kann so folgende konkrete Problemstellung
abgeleitet werden.
\begin{center}
	\textit{"Welche Analyse-Erweiterungen gibt es und welche Möglichkeiten ergeben
	sich durch einen Einsatz dieser Konstrukte."}
\end{center}
Sofern auf die Forschungsfrage eine Antwort gefunden werden kann, besteht am Ende
dieser Arbeit ein grober Überblick über die bekanntesten Analyse-Erweiterungen im
SQL-Standard und für welche Fälle diese eingesetzt werden. Außerdem soll ein
Verständnis bestehen, wie auch komplexere Datenabfragen durch Einsatz der Analyse-Erweiterungen
vereinfacht werden. Darüber hinaus soll die Arbeite eine praxisnahe
Orientierungshilfe liefern, die als Stütze bei der Datenanalyse mit SQL fungiert.
Um diesen Aspekt noch weiter zu verdeutlichen, soll das nächste Kapitel die Relevanz
dieser ARbeit noch genauer beleuchten.
% --------------------------------------------------------------------------------------------

\section{Relevanz der Analyse-Erweiterungen}
\label{sec:relevant} Datenanalysen gehören heute zu den unverzichtbaren Werkzeugen
für Entscheidungen in Unternehmen und Forschung. Da sich viele Unternehmen auf
die klassischen relationalen Datenbanken stützen, wird SQL als Standard zur Abfrage
von Daten und zur Erstellung von Analysen eingesetzt. Die SQL Analyse-Erweiterungen
sollen genau diesen Analyseprozess unterstützen.

Die Analyse-Erweiterungen für SQL finden vorrangig ihren Einsatz im Bereich Data
Analysis, so belegen es auch \citet[Kapitel 3]{FOTACHE2015243}. Es ist in diesem
Kontext oft von \textit{analytical} oder \textit{window functions} die Rede
\citep[vgl.][Kapitel 3]{FOTACHE2015243}. Die Begriffe \textit{statistical
inspired aggregate functions} und \textit{multiple group by operators} tauchen im
Fachbereich Data Analysis ebenfalls immer wieder auf \citep[vgl.][Kapitel 4.3]{FOTACHE2015243}.
Alle diese Fachbegriffe sind auf die Analyse-Erweiterungen in SQL zurückzuführen.
Auch in den Punkten Komplexität und Kompaktheit können die Erweiterungen punkten.
Mit den Analyse-Erweiterungen ist eine komplexe Datenanalyse durch nur wenige
Zeilen Quelltext möglich so \citet[Abstract]{Maue2022}.

Es wird also deutlich, dass die Analyse-Erweiterungen nicht nur eine komplexe Analyse
zulassen, sondern für das ganze Gebiet der Datenanalyse einen hohen Wert bietet.
Sie gewähren eine weitaus detailliertere Einsicht in Datensätze, als es mit
herkömmlichen Befehlen möglich ist. Daraus lässt sich auch ableiten, dass das
Themer der Analyse-Erweiterungen große Kreise wirft. Um beim lesen dieser Arbeit
einen groben Überblick zu schaffen wird die Gliederung der Arbeit im Folgenden Besxchrieben
% --------------------------------------------------------------------------------------------

\section{Aufbau der Arbeit}
\label{sec:aufbau_der_arbeit} Diese Arbeit liefert am Ende einen groben
Überblick über die Analyse-Erweiterungen und deren Möglichkeiten. Darüber hinaus
sollen einige praxisbezogenen Beispiele diskutieren werden. Hierzu teilt sich
das Thema in drei Teile auf. Der erste Teil bildet eine Recherche zum aktuellen Stand
der Technik und führt in die Grundlagen der hier fokussierten Erweiterungen ein.
Hierzu wird einen Blick auf die Funktionsweise der unterschiedlichen
Erweiterungen geworfen. Ergänzend dazu sollen auch zusätzliche Erweiterungen aufgezählt
werden. Der zweite Teil bildet den Kern der Arbeit und soll die fokussierten
Analyse-Erweiterungen anhand eines passenden Beispiels erläutern. Hierzu wird ein
globales Beispiel einer relationalen Datenbank generiert, mit deren Hilfe alle
der ausgewählten Erweiterungen demonstriert und analysiert werden. Für die Analyse
wird auf Bewertungskriterien zurückgegriffen, die generisch gewählt werden. So können
die sehr unterschiedlichen Erweiterungen kategorisiert werden. Im letzten Teil
der Arbeit sollen die Ergebnisse evaluiert werden. Hierzu werden die Beispiele analysiert
und mögliche Interpretationen ausgearbeitet.
% --------------------------------------------------------------------------------------------

\section{Fokus der Arbeit}
\label{sec:fokus_der_arbeit} Es existieren diverse Analyse-Erweiterungen für SQL
die für die unterschiedlichsten Anwendungsfälle eingesetzt werden können. Einige
dieser Erweiterungen lösen Nischenprobleme, andere bieten eine breite Palette an
Funktionen. Dieser Artikel wählt zwei der vielen Analyse-Erweiterungen aus und
setzt so einen Fokus.

Für eine konkrete Auswahl stützt sich die Arbeit auf Erweiterungen, die laut \citet[Kapitel3]{FOTACHE2015243},
für den Fachbereich Data Analysis eine wichtige Rolle spielen. \citet[Kapitel 3]{FOTACHE2015243}
sprechen in ihrer Arbeit von zwei wichtigen Punkten, die auch hier den Fokus
bilden sollen.

Wie bereits zu Anfang der Arbeit hervorgeht, sind zwei dieser möglichen Erweiterungen
die folgenden.

\begin{description}
	\item[$\bullet$ Window-Functions] als statistical inspired aggregate functions
		\\ \citep[vgl.][Kapitel 4.3]{FOTACHE2015243}

	\item[$\bullet$ ROLLUP and CUBE] als multiple group by operators \\ \citep[vgl.][Kapitel4.3]{FOTACHE2015243}
\end{description}

Im Laufe des Artikels soll Wissen zu diesen beiden Erweiterungen gesammelt und
an praxisbezogenen Beispiel angewandt werden. Alle weiteren Analyse-Erweiterungen
werden in diesem Artikel nur kurz namentlich erwähnt und erfahren keine weitere Behandlung.
Um dies zu gewährleisten sind einige Theoretische Grundlagen nötig, die zum verstehen
der Ergebnisse essenziell sind. Das Kapitel \ref{chap:hintergund_und_grundlagen}
führt demnach in alle wichtigen Hintergründe ein.
	\chapter{Hintergrund}
Dieses Kapitel soll die Grundsteine legen und in die Konzepte und Theorien
der Analyse-Erweiterungen einführen. Um die Fragestellung dieser Arbeit einordnen zu
können, ist unter anderem  ein Blick auf die Historie hilfreich. Der Hauptgrund für
die Wahl der drei oben genannten Erweiterungen liegt nämlich nicht nur in ihren vielseitigen
Möglichkeiten, sondern auch in ihrer Integration in den SQL-Standard, wie auch von
\cite{grust2017advanced} in seiner Arbeit hervorgehoben wird. Er verdeutlicht auf Seite 10,
dass im Jahre 1987 die erste Version des SQL-Standard veröffentlicht wurde. Mit SQL3 1999
kamen dann die ersten Analyse-Erweiterungen mit den \textit{recursive queries} hinzu.
Wie \cite{melton2001sql} in Kapitel 9.12 ihrer Arbeit beschreiben, wurden die Funktionen
\textit{ROLLUP} und \textit{CUBE} ebenfalls 1999 in den SQL-Standard aufgenommen.
Die \textit{Window Functions} hingegen folgten laut \cite{grust2017advanced} erst
mit der späteren Version im Jahr 2003.









\section{SQL Kurzüberischt}

\section{Window Funktionen}

\subsection{Definition}

\subsection{Aufbau}

\section{Common Table Expressions und rekursive Queries}

\subsection{Common Table Expressions (CTE)}

\subsection{Rekursive Queries}

\section{Rollup and Cube}

\subsection{Grouping Sets}

\subsection{Rollup}

\subsection{Cube}

\section{Dialekte und Unterstüzung}
	\chapter{Methodik}
In dieses Kapitel sollen die fokusierten Analyse-Erweiterungen genauer unter die
Lupe genommen werden. Hierzu werden praxisnahe Beispiele für die Erweiterungen
genneriert und hernagezogen. Nachdem eine Einblick in die Beispiele gewonnen wurde,
soll anhand von speziellen Bewertungskriterien die Sprachkonstrukte und
Möglichleiten dieser Analyse-Erweiterungen objektiv bewertet werden. Auch die Ergebnismenge
der verschiednen Beispiel soll in die Bewertung mit einfließen.

Für sollch ein Vorgehen, wird im folgenden Kapitel der entsprechende Rahmen
gesetzt.

\section{Rahmenbedingungen}
Befor mit dem erstellen einer Beispieldatenbank begonnen werden kann, muss ein Datenbank
Management System (DBMS) gewählt werden. Aufgrund der einfachen Installation
wird für die folgenden Beispiele eine MYSQL Datenbank verwendet. Dieses bietet alle
Erweiterungen, die in dieser Arbeit näher betrachtet werden sollen.

Als praxisnahes Beispiel wird auf ein allgemein bekanntes konzept
zurückgegriffen um die Komplexität zu reduzieren und den Fokus mehr auf die
Analyseerweiterungen zu lenken. Im Rahmen dieses Artikel soll das Konzept einer Bibliothek
als Beispiel fungieren. Dieses trägt vortan den Namen \textit{LibraryDB}. Das
Datenmodell teilt sich hier in drei Relationen auf.
\begin{description}
	\item[$\bullet$ 1. Relation] Books
	\item[$\bullet$ 2. Relation] Members
	\item[$\bullet$ 3. Reltaion] BorrowedBooks
\end{description}
Die Relation \textit{BorrowedBooks} stellt eine Verbindung zwischen den Tabellen
\textit{Members} und \textit{Books} her. So kann Beispielsweise auch ein Mitglied
mehrere Bücher ausleihen. Die folgende Abbildung zeigt das genaue logische Datenmodell.

TABELLEN einbauen

FÜr eine objektive und Einheitliche Bewertung der Beispiele sollen zuletzt noch Parameter
festlgelegt weden, anhand deren eine Beurteilung möglich ist. Für Diese Arbeiten werden
folgende Kriterien herangezogen:
\begin{description}
	\item[$\bullet$ 1. Parameter] Lesbarkeit
	\item[$\bullet$ 2. Parameter] Flexibilität
	\item[$\bullet$ 3. Parameter] Ergebnismenge
\end{description}
Im ersten Punkt soll ein Blick auf 


\section{Beispiel analytische Funktion}

\section{Beispiel - Rekursive Queries}

\section{Beispiel - Rollup and Cube}
	\chapter{Ergebnis}

\section{Analyse}

\section{Interpretation}
	\chapter{Fazit}
Testeinleitung

\section{Sektion1}
Testsektion \cite{adobe}

	%%% --------------------------------------------------
	%%% Verzeichnisse
	%%% --------------------------------------------------

	\listoffigures % Abbildungsverzeichnis
	\addcontentsline{toc}{chapter}{Abbildungsverzeichnis}
	\listoftables % Tabellenverzeichnis
	\addcontentsline{toc}{chapter}{Tabellenverzeichnis}
	\renewcommand{\lstlistlistingname}{Quellcodeverzeichnis}
	\lstlistoflistings % Listings
	\addcontentsline{toc}{chapter}{Quellcodeverzeichnis}

	% --------------------------------------------------
	% Bibliographie
	% --------------------------------------------------
	\renewcommand{\bibfont}{\footnotesize}
	\printbibliography
	[title={Literaturverzeichnis}, heading=bibintoc]

	%%% --------------------------------------------------
	%%% Anhang
	%%% --------------------------------------------------
	\appendix

	\addcontentsline{toc}{chapter}{SQL-Skript LibraryDB}
\label{sec:library_db}
\textbf{Hinweis:} Dieses SQL-Skript bildet die Grundlage für die in der
Arbeit gezeigten Beispiele.

\begin{lstlisting}[caption={SQL Skript LibraryDB}, label={list:skript_library_db}]
-- Create the LibraryDB database
CREATE DATABASE LibraryDB;
USE LibraryDB;
-- Creates the table books
CREATE TABLE Books (
    book_id INT AUTO_INCREMENT PRIMARY KEY,
    title VARCHAR(255) NOT NULL,
    author VARCHAR(255) NOT NULL,
    genre VARCHAR(50),
    published_year INT CHECK (published_year > 0)
);
-- Creats the table Members
CREATE TABLE Members (
    member_id INT AUTO_INCREMENT PRIMARY KEY,
    name VARCHAR(255) NOT NULL,
    membership_start_date DATE,
    membership_type ENUM('Standard', 'Premium')
        DEFAULT 'Standard',
    age_group ENUM('Child', 'Teen', 'Adult')
        NOT NULL
);
-- Creats the table Borrowed_Books
CREATE TABLE Borrowed_Books (
    borrow_id INT AUTO_INCREMENT PRIMARY KEY,
    member_id INT,
    book_id INT,
    borrow_date DATE,
    return_date DATE,
    FOREIGN KEY (member_id)
        REFERENCES Members(member_id) ON DELETE CASCADE,
    FOREIGN KEY (book_id)
        REFERENCES Books(book_id) ON DELETE CASCADE
);
\end{lstlisting}
	\addcontentsline{toc}{section}{Beispieldaten LibraryDB}
\section*{Beispieldaten LibraryDB}
\label{sec:library_db_sample_data} \textbf{Hinweis:} Dieser Datensatz dient lediglich
als Beispiel und bildet nicht die vollständig Grundlage für die in der Arbeit gezeigten
Beispiele.

\begin{lstlisting}[
    caption={Beispieldaten für LibraryDB Books},
    label={list:sample_data_library_db}]
-- Sample data for table Books
INSERT INTO Books (title, author, genre, published_year)
VALUES
('To Kill a Mockingbird', 'Harper Lee', 'Fiction', 1960),
('The Great Gatsby', 'F. Scott Fitzgerald', 'Fiction', 1925),
('The Catcher in the Rye', 'J.D. Salinger', 'Fiction', 1951),
('Moby-Dick', 'Herman Melville', 'Adventure', 1851),
('Pride and Prejudice', 'Jane Austen', 'Romance', 1813),
('War and Peace', 'Leo Tolstoy', 'Historical', 1869),
('The Hobbit', 'J.R.R. Tolkien', 'Fantasy', 1937),
('The Da Vinci Code', 'Dan Brown', 'Thriller', 2003),
('The Catch-22', 'Joseph Heller', 'Satire', 1961),
('The Shining', 'Stephen King', 'Horror', 1977);
\end{lstlisting}

\begin{lstlisting}[
    caption={Beispieldaten für LibraryDB Member},
    label={list:sample_data_library_db_Member}]
-- Sample data for table Members
INSERT INTO Members (name, membership_start_date, membership_type, age_group)
VALUES
('Alice Smith', '2020-01-15', 'Premium', 'Adult'),
('Bob Johnson', '2021-06-10', 'Standard', 'Teen'),
('Charlie Brown', '2019-04-05', 'Standard', 'Child'),
('Diana Prince', '2021-09-25', 'Premium', 'Adult'),
('Eve White', '2020-11-17', 'Standard', 'Teen'),
('Frank Miller', '2022-02-20', 'Standard', 'Adult'),
('Grace Hopper', '2021-03-15', 'Premium', 'Adult'),
('Irene Adler', '2020-12-01', 'Standard', 'Adult'),
('James Moriarty', '2019-07-25', 'Premium', 'Teen'),
('Lucy Gray', '2021-08-30', 'Standard', 'Teen'),
('Michael Scott', '2022-06-10', 'Premium', 'Adult');
\end{lstlisting}

\filbreak

\begin{lstlisting}[
    caption={Beispieldaten für LibraryDB:Borrowd Books},
    label={list:sample_data_library_db_borrow}]
-- Sample data table Borrowd_Books
-- Null means not yet returned
INSERT INTO Borrowed_Books
(member_id, book_id, borrow_date, return_date)
VALUES
(1, 1, '2023-01-01', '2023-01-15'),
(2, 2, '2023-02-01', '2023-02-10'),
(3, 3, '2023-03-05', '2023-03-20'),
(1, 4, '2023-04-10', NULL),
(4, 5, '2023-05-12', '2023-06-01'),
(2, 1, '2023-06-15', NULL);
\end{lstlisting}

	\AuthorDeclaration % Selbständigkeitserklärung

	% --------------------------------------------------
	% Index
	% --------------------------------------------------
	{\setkomafont{section}{\Huge} % temporarily set chapter font
	\printindex }
\end{document}