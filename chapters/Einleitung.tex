\chapter{Einleitung}
der SQL-Standard wird bald 40 Jahre alt und hat bis heute diverse
Erweiterungen erfahren. Ein der wichtigsten sind die Analyse-Erweiterungen. Einige
dieser Analyse-Fähigkeiten haben es im Laufe der Jahre in den SQL-Standard
geschafft und sind von dortan ein fester Bestandteil von SQL. Trotz Ihrer doch
langen Geschichte, stehen die Analyse-Erweiterungen eher im Schaten der breiten
Palette an SQL-Befehlen und konnten nicht sonderlich viel an Bekanntheit gewinnen.


\section{Ziel der Arbeit}
Dieser Artikel soll nun Licht ins Dunkle bringen und das Geheimnis dahinter lüften.
Es werden einige dieser Analyse-Erweiterungen an Hand einfacher Beispiele aufzeigen,
um Sie als Leserin oder Leser auf den Geschack zu bringen. Hierzu sollen die
unterschiedlichen Sprachkonstukte und Methoden demonstriert und die Konzepte
dahinter erläutert werden. Es kann so diese konkrete Problemstellung abgeleitet
werden:

\textbf{Problemstellung:}
\begin{center}
    \textit{"Welche Analyse-Erweiterungen gibt es und welche Möglichkeiten ergebn sich durch
    einen Einsatz dieser Konstrukte."}
\end{center}

Sofern auf die Forschungsfrage eine Antwort gefunden weden kann, besteht am Ende
dieser Arbeit ein grober Überblick über die bekanntesten Analyse-Erweiterungen im
SQL-Standard und für welche Fälle diese eingesetzt werden. Außerdem soll
ein Verständnis bestehen, wie auch komplexere Datenabfragen durch einsatz der
Analyse-Erweiterungen vereinfacht weden könne.

Darüberhinaus soll die Arbeite eine praxisnahe Orientierungshilfe liefern, die als
Stütze bei der Datenanalyse mit SQL fungiert.










\begin{lstlisting} [caption={This is a code block}]
    -- This is a comment
    SELECT name, age, AVG(age) AS avgAge
    FROM users
    WHERE name = 'Heiko'
    GROUP BY city
\end{lstlisting}




\section{Relevanz der Analyse-Erweiterungen}
Datenanalysen gehören heute zu den unverzichtbaren Werkzeugen für Entscheidungen
in Unternehmen und Forschung. Da sich viele Unternehmen auf die klassischen
relationalen Dantenbanken stützen, wird SQL als Standard zur Abfrage von Daten und
zur ersetllung von Analysen eingesetzt. Die Analye-Erweiterungen, welche in dieser
Arbeit behandelt weden sollen sind hierfür eine sehr mächtiges Hilfsmittel.


\section{Aufbau der Arbeit}
