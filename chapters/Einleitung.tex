\chapter{Einleitung}
Der SQL-Standard wird bald 40 Jahre alt und hat bis heute diverse Erweiterungen erfahren.
Ein der wichtigsten sind die Analyse-Erweiterungen. Einige dieser Analyse-Fähigkeiten
haben es im Laufe der Jahre in den SQL-Standard geschafft und sind ab diesem Zeitpunkt
ein fester Bestandteil von SQL. Trotz Ihrer doch langen Geschichte, stehen die Analyse-Erweiterungen
eher im Schatten der breiten Palette an SQL-Befehlen und konnten nicht sonderlich
viel an Bekanntheit gewinnen.

\section{Ziel der Arbeit}
Dieser Artikel soll für Klarheit sorgen und das Geheimnis dahinter offenbaren.
Es werden einige dieser Analyse-Erweiterungen anhand einfacher Beispiele
aufzeigen, um eine eingehende Erklärung zu bieten. Hierzu sollen die
unterschiedlichen Sprachkonstrukte und Methoden demonstriert und die Konzepte
dahinter erläutert werden. Es kann so folgende konkrete Problemstellung abgeleitet
werden:

\begin{center}
	\textit{"Welche Analyse-Erweiterungen gibt es und welche Möglichkeiten ergeben
	sich durch einen Einsatz dieser Konstrukte."}
\end{center}

Sofern auf die Forschungsfrage eine Antwort gefunden werden kann, besteht am Ende
dieser Arbeit ein grober Überblick über die bekanntesten Analyse-Erweiterungen im
SQL-Standard und für welche Fälle diese eingesetzt werden. Außerdem soll ein
Verständnis bestehen, wie auch komplexere Datenabfragen durch Einsatz der Analyse-Erweiterungen
vereinfacht werden.

Darüber hinaus soll die Arbeite eine praxisnahe Orientierungshilfe liefern, die
als Stütze bei der Datenanalyse mit SQL fungiert.

\section{Relevanz der Analyse-Erweiterungen}
Datenanalysen gehören heute zu den unverzichtbaren Werkzeugen für Entscheidungen
in Unternehmen und Forschung. Da sich viele Unternehmen auf die klassischen relationalen
Datenbanken stützen, wird SQL als Standard zur Abfrage von Daten und zur
Erstellung von Analysen eingesetzt. Die SQL Analyse-Erweiterungen sollen genau diesen
Analyseprozess unterstützen.

Die Analyse-Erweiterungen für SQL finden vorrangig ihren Einsatz im Bereich Data
Analysis (\cite{FOTACHE2015243}, Kapitel 3). Es ist in diesem Kontext oft von \textit{analytical}
oder \textit{window functions} die Rede (\cite{FOTACHE2015243}, Kapitel 3). Die Begriffe
\textit{statistical inspired aggregate functions} und \textit{multiple group by
operators} tauchen im Fachbereich Data Analysis ebenfalls immer wieder auf (\cite{FOTACHE2015243},
Kapitel 4.3). Alle diese Fachbegriffe sind auf die Analyse-Erweiterungen in SQL zurückzuführen.
Auch in den Punkten Komplexität und Kompaktheit können die Erweiterungen punkten.
Mit den Analyse-Erweiterungen ist eine komplexe Datenanalyse durch nur wenige Zeilen
Quelltext möglich (\cite{Maue2022}, Abstract).

Es wird also deutlich, dass die Analyse-Erweiterungen nicht nur eine komplexe Analyse
zulassen, sondern für das ganze Gebiet der Datenanalyse einen hohen Wert bietet.
Sie gewähren eine weitaus detailliertere Einsicht in Datensätze, als es mit
herkömmlichen Befehlen möglich ist.

\section{Aufbau der Arbeit}
Diese Arbeit liefert am Ende einen groben Überblick über die Analyse-Erweiterungen
und deren Möglichkeiten. Darüber hinaus sollen einige praxisbezogenen Beispiele diskutieren
werden. Hierzu teilt sich das Thema in drei Teile auf.

Der erste Teil bildet eine Recherche zum aktuellen Stand der Technik und führt
in die Grundlagen der hier fokussierten Erweiterungen ein. Hierzu wird einen Blick
auf die Funktionsweise der unterschiedlichen Erweiterungen geworfen. Ergänzend
dazu sollen auch zusätzliche Erweiterungen aufgezählt und kurz angeschnitten
werden.

Der zweite Teil bildet den Kern der Arbeit und soll die fokussierten Analyse-Erweiterungen
anhand eines passenden Beispiels erläutern. Hierzu wird ein globales Beispiel einer
relationalen Datenbank generiert, mit deren Hilfe alle der ausgewählten
Erweiterungen demonstriert und analysiert werden. Für die Analyse wird auf Bewertungskriterien
zurückgegriffen, die generisch gewählt werden. So können die sehr unterschiedlichen
Erweiterungen kategorisiert werden.

Im letzten Teil der Arbeit sollen die Ergebnisse evaluiert werden. Hierzu werden
die Beispiele analysiert und mögliche Interpretationen ausgearbeitet. Die zu späterem
Zeitpunkt definierten Bewertungskriterien spielen hierbei eine wesentliche Rolle.

\section{Fokus der Arbeit}
Es existieren diverse Analyse-Erweiterungen für SQL die für die unterschiedlichsten
Anwendungsfälle eingesetzt werden können. Einige dieser Erweiterungen lösen
Nischenprobleme, andere bieten eine breite Palette an Funktionen. Dieser Artikel
wählt zwei der vielen Analyse-Erweiterungen aus und setzt so einen Fokus.

Für eine konkrete Auswahl stützt sich die Arbeit auf Erweiterungen, die laut \citet{FOTACHE2015243},
Kapitel 3 für den Fachbereich Data Analysis eine wichtige Rolle spielen. Wie
bereits zu Anfang der Arbeit hervorgeht, sind zwei dieser möglichen Erweiterungen
die folgenden.
\begin{description}
	\item[$\bullet$ Window-Functions] als Erweiterung der klassischen Aggregatfunktionen
		\\ (\cite{FOTACHE2015243}, Kapitel 4.3)

	\item[$\bullet$ ROLLUP and CUBE] als Erweiterung der GROUP By-Klausel \\ (\cite{FOTACHE2015243},
		Kapitel 4.3)
\end{description}
Im Laufe des Artikels soll Wissen zu diesen beiden Erweiterungen gesammelt und
an praxisbezogenen Beispiel angewandt werden. Alle weiteren Analyse-Erweiterungen
werden in diesem Artikel nur namentlich erwähnt und erfahren keine weitere Behandlung.