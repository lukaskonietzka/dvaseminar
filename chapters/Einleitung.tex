\chapter{Einleitung}
Der SQL-Standard wird bald 40 Jahre alt und hat bis heute diverse Erweiterungen erfahren.
Ein der wichtigsten sind die Analyse-Erweiterungen. Einige dieser Analyse-Fähigkeiten
haben es im Laufe der Jahre in den SQL-Standard geschafft und sind ab diesem Zeitpunkt
ein fester Bestandteil von SQL. Trotz Ihrer doch langen Geschichte, stehen die Analyse-Erweiterungen
eher im Schatten der breiten Palette an SQL-Befehlen und konnten nicht sonderlich
viel an Bekanntheit gewinnen.

\section{Ziel der Arbeit}
Dieser Artikel soll nun Licht ins Dunkle bringen und das Geheimnis dahinter lüften.
Es werden einige dieser Analyse-Erweiterungen anhand einfacher Beispiele
aufzeigen, um Sie als Leserin oder Leser auf den Geschmack zu bringen. Hierzu
sollen die unterschiedlichen Sprachkonstrukte und Methoden demonstriert und die
Konzepte dahinter erläutert werden. Es kann so diese konkrete Problemstellung abgeleitet
werden:

\begin{center}
	\textit{"Welche Analyse-Erweiterungen gibt es und welche Möglichkeiten ergeben
	sich durch einen Einsatz dieser Konstrukte."}
\end{center}

Sofern auf die Forschungsfrage eine Antwort gefunden werden kann, besteht am Ende
dieser Arbeit ein grober Überblick über die bekanntesten Analyse-Erweiterungen im
SQL-Standard und für welche Fälle diese eingesetzt werden. Außerdem soll ein
Verständnis bestehen, wie auch komplexere Datenabfragen durch Einsatz der Analyse-Erweiterungen
vereinfacht werden könne.

Darüber hinaus soll die Arbeite eine praxisnahe Orientierungshilfe liefern, die
als Stütze bei der Datenanalyse mit SQL fungiert.

\section{Relevanz der Analyse-Erweiterungen}
Datenanalysen gehören heute zu den unverzichtbaren Werkzeugen für Entscheidungen
in Unternehmen und Forschung. Da sich viele Unternehmen auf die klassischen relationalen
Datenbanken stützen, wird SQL als Standard zur Abfrage von Daten und zur
Erstellung von Analysen eingesetzt. Die SQL Analyse-Erweiterungen sollen genau diesen
Analyseprozess unterstützen.

\cite{FOTACHE2015243} schrieben hierzu in ihrem Artikel über die Verwendung von SQL
im Bereich Data Science. Bereits dort erwähnten sie einige der Analyse-Erweiterungen,
die sich unterstützend für die Datenanlyse auswirken. In Kapitel 3 ihrer
Veröffentlichung (SQL Features for Data Science) schrieben sie von \textit{analytical
processing} oder \textit{window functions}. Des Weiteren wird in Kapitle 4.3 (OLAP
and SQL) unter anderem von \textit{statistical inspired aggregate functions} und
\textit{multiple group by operators} gesprochen. Einen Weiteren Indiz für den
Einsatz der Analyse-Erweiterungen im Bereich der Datenanalyse bietet
\cite{Maue2022} in seinem Internetartikel, wo er kurz darüber Berichte, welchen
Mehrwert die Analyse-Funktionen haben. Er schrieb hier von einer komplexen Datenanalyse
durch nur wenige Zeilen Quelltext.

Es wird also deutlich, dass die Analyse-Erweiterungen nicht nur eine komplexe
Analyse zulassen, sondern für das ganze Gebiet der Datenanalyse einen hohen Wert
bietet. Sie gewähren eine weitaus detailliertere Einsicht in Datensätze, als es mit
herkömmlichen Befehlen möglich ist.

\section{Aufbau der Arbeit}
Wie in Kapitel 2.1 beschrieben, soll diese Arbeit am Ende einen groben Überblick
über die Analyse-Erweiterungen und deren Möglichkeiten geben. Darüber hinaus sollen
einige praxisbezogenen Beispiele diskutieren werden. Hierzu teilt sich das Thema
in drei Teile auf.

Der Erste Teil soll den Einstieg erleichtern und bietet einige Grundlagen zum
Inhalt. Hierzu wird im ersten Schritt auf SQL Grundlagen eingegangen, die zum
Verstehen des eigentlichen Kerns der Arbeit notwendig sind. Darüber hinaus führen
wir die konkreten Erweiterungen ein, die wir hier näher betrachten werden. Hierzu
werfen wir einen Blick auf die Funktionsweise der unterschiedlichen Erweiterungen.
Ergänzend dazu sollen auch zusätzliche Erweiterungen erwähnt werden.

Der zweite Teil bildet den Kern der Arbeit und soll die fokussierten Analyse-Erweiterungen
anhand eines passenden Beispiels erläutern. Hierzu wird ein globales Beispiel einer
relationalen Datenbank generiert, mit deren Hilfe alle der ausgewählten
Erweiterungen demonstriert und analysiert werden. Für die Analyse wird auf Bewertungskriterien
zurückgegriffen, die generisch gewählt werden. So können die sehr unterschiedlichen
Erweiterungen kategorisiert werden.

Im letzten Teil der Arbeit sollen die Ergebnisse evaluiert werden. Hierzu werden
die Beispiele analysiert und mögliche Interpretationen ausgearbeitet. Die zu Anfang
definierten Bewertungskriterien spielen hierbei eine wesentliche Rolle.

\section{Fokus der Arbeit}
Es existieren diverse Analyse-Erweiterungen für SQL die für die unterschiedlichsten
Anwendungsfälle eingesetzt werden können. Einige dieser Erweiterungen lösen
Nischenprobleme, andere bieten eine breite Palette an Funktionen. Dieser Artikel
wählt drei der vielen Analyse-Erweiterungen aus und setzt so einen Fokus.

Für die Auswahl dieser konkreten Erweiterungen stützt sich die Arbeit auf das
Paper von \cite{FOTACHE2015243} die im Kontext Data Science schon einige
Erweiterungen erwähnt haben. Wie bereits zu Anfang der Arbeit beschrieben sind zwei
dieser Erweiterungen die folgenden:

\begin{description}
	\item[$\bullet$ Window-Functions] als Erweiterung der klassischen Aggregatfunktionen

	\item[$\bullet$ ROLLUP and CUBE] als Erweiterung der GROUP By-Klausel
\end{description}

Für die dritte Erweiterung, die hier fokussiert werden soll, sei auf den Artikel
von \cite{4460710} verwiesen. Bereits im ersten Kapitel wird deutlich, dass
Rekursion ein fester Bestandteil der Informatik ist. Er betont, dass viele
Algorithmen auf Basis der Rekursion definiert sind. Mit SQL3 (1999) wird die Rekursion
auch in den SQL-Standard eingeführt und soll hier die dritte Erweiterung sein,
die behandelt wird.

\begin{description}
	\item[$\bullet$ Recursive Queries] als Erweiterung der Commen Table Extentions
		(CTE)
\end{description}

Im Laufe des Artikels soll Wissen zu diesen drei Erweiterungen gesammelt und an praxisbezogenen
Beispiel angewandt werden. Alle weiteren Analyse-Erweiterungen werden in diesem
Artikel nur namentlich erwähnt und erfahren keine weitere Behandlung.