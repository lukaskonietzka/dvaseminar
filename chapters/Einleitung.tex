\chapter{Einleitung}
\label{chap:einleitung} Der SQL-Standard wird bald 40 Jahre alt und hat bis heute
diverse Erweiterungen erfahren. Ein der Wichtigsten sind die Analyse-Erweiterungen.
Einige dieser Erweiterungen haben es im Laufe der Jahre in den SQL-Standard geschafft
und sind ab diesem Zeitpunkt ein fester Bestandteil. Trotz Ihrer doch langen
Geschichte, stehen die Analyse-Erweiterungen eher im Schatten der breiten Palette
an SQL-Befehlen und konnten nicht sonderlich viel an Bekanntheit gewinnen.
% --------------------------------------------------------------------------------------------

\section{Ziel der Arbeit}
\label{sec:ziel_der_arbeit} Die vorliegende Arbeit soll für Klarheit sorgen und das
Geheimnis dahinter auzeigen. Es werden einige dieser Analyse-Erweiterungen
anhand einfacher Beispiele aufzeigen, um eine eingehende Erklärung zu bieten.
Hierzu sollen die unterschiedlichen Sprachkonstrukte und Methoden demonstriert
und die Konzepte dahinter erläutert werden. Es kann so folgende konkrete Problemstellung
abgeleitet werden.
\begin{center}
	\textit{"Welche Analyse-Erweiterungen gibt es und welche Möglichkeiten ergeben
	sich durch einen Einsatz dieser Konstrukte."}
\end{center}
Sofern auf die Forschungsfrage eine Antwort gefunden werden kann, besteht am Ende
dieser Arbeit ein grober Überblick über die bekanntesten Analyse-Erweiterungen im
SQL-Standard und für welche Fälle diese eingesetzt werden. Außerdem soll ein
Verständnis geschaffen werden, wie auch komplexere Datenabfragen durch Einsatz der
Analyse-Erweiterungen vereinfacht werden. Darüber hinaus soll die Arbeite eine
praxisnahe Orientierungshilfe liefern die als Stütze bei der Datenanalyse mit
SQL fungiert. Um diesen Aspekt noch weiter zu verdeutlichen, soll das nächste
Kapitel die Relevanz dieser Arbeit noch genauer beleuchten.
% --------------------------------------------------------------------------------------------

\section{Relevanz der Analyse-Erweiterungen}
\label{sec:relevant} Datenanalysen gehören heute zu den unverzichtbaren
Werkzeugen für Entscheidungen in Unternehmen und Forschung. Da sich viele Unternehmen
auf die klassischen relationalen Datenbanken stützen, wird SQL als Standard zur
Abfrage von Daten und zur Erstellung von Analysen eingesetzt. Die SQL Analyse-Erweiterungen
sollen genau diesen Analyseprozess unterstützen.

Die Analyse-Erweiterungen für SQL finden vorrangig ihren Einsatz im Bereich Data
Analysis, so belegen es auch \citet[Kapitel 3]{FOTACHE2015243}. Es ist in diesem
Kontext oft von \textit{analytical} oder \textit{window functions} die Rede \citep[vgl.][Kapitel
3]{FOTACHE2015243}. Die Begriffe \textit{statistical inspired aggregate
functions} und \textit{multiple group by operators} tauchen im Fachbereich Data
Analysis ebenfalls immer wieder auf \citep[vgl.][Kapitel 4.3]{FOTACHE2015243}. Alle
diese Fachbegriffe sind auf die Analyse-Erweiterungen in SQL zurückzuführen.
Auch in den Punkten Komplexität und Kompaktheit können die Erweiterungen punkten.
Mit den Analyse-Erweiterungen ist eine komplexe Datenanalyse durch nur wenige Zeilen
Quelltext möglich, so \citet[vgl.][]{Maue2022}.

Es wird also deutlich, dass die Analyse-Erweiterungen nicht nur eine komplexe
Analyse zulassen, sondern für das ganze Gebiet der Datenanalyse einen hohen Wert
bietet. Sie gewähren eine weitaus detailliertere Einsicht in Datensätze als es
mit herkömmlichen Befehlen möglich ist. Daraus lässt sich auch ableiten, dass
das Thema der Analyse-Erweiterungen große Kreise schlägt. Dieses umfangreiche Gebiet
bedarf einer kurzen Gliederung, die hier eingeleitet werden soll.
% --------------------------------------------------------------------------------------------

\section{Aufbau der Arbeit}
\label{sec:aufbau_der_arbeit} Für diese Arbeit erfolgt eine Unterteilung in fünf
Hauptkapitel, die eine systematische Annäherung an das Thema der Analyse-Erweiterungen
ermöglichen. Nach der Einleitung, in der die Zielsetzung, Relevanz und der Fokus
der Arbeit dargelegt werden, folgt das Kapitel Theoretische Grundlagen. Hier
werden die wesentlichen Konzepte, wie \textit{Window Functions} und
multidimensionale \textit{Group By Klauseln}, ausführlich behandelt. Das dritte Kapitel,
Methodik, beschreibt die Vorgehensweise der Arbeit. Es umfasst die durchgeführte
Literaturrecherche und die Implementierungsschritte, die die Grundlage für die später
präsentierten Ergebnisse bilden. Im vierten Kapitel, Ergebnisse, werden die
Resultate der Implementierung dargestellt. Dabei liegt der Fokus auf der praktischen
Anwendung der behandelten Konzepte, ergänzt durch zusätzliche Erweiterungen, die
den Nutzen der unterschiedlichen Erweiterungen verdeutlichen. Abschließend wird
im Kapitel Fazit eine Zusammenfassung der gewonnenen Erkenntnisse gegeben und
ein Ausblick auf mögliche zukünftige Arbeiten präsentiert. Ergänzend dazu enthält
der Anhang die SQL-Skripte und Beispieldaten, die zur Reproduzierbarkeit der
Ergebnisse beitragen.
% --------------------------------------------------------------------------------------------

\section{Fokus der Arbeit}
\label{sec:fokus_der_arbeit} Es existieren diverse Analyse-Erweiterungen für SQL
die für die unterschiedlichsten Anwendungsfälle eingesetzt werden können. Einige
dieser Erweiterungen lösen Nischenprobleme, andere bieten eine breite Palette an
Funktionen. Diese Arbeit wählt zwei der vielen Analyse-Erweiterungen aus und setzt
so einen Fokus.

Für eine konkrete Auswahl stützt sich die Arbeit auf Erweiterungen, die laut
\citet[Kapitel3]{FOTACHE2015243}, für den Fachbereich Data Analysis eine wichtige
Rolle spielen. \citet[Kapitel 3]{FOTACHE2015243} sprechen in ihrer Arbeit von
zwei wichtigen Punkten, die auch hier den Fokus bilden sollen.

\begin{description}
	\item[$\bullet$ Window-Functions] - als statistical inspired aggregate functions
		\\ \citep[vgl.][Kapitel 4.3]{FOTACHE2015243}

	\item[$\bullet$ ROLLUP and CUBE] - als multiple group by operators \\ \citep[vgl.][Kapitel4.3]{FOTACHE2015243}
\end{description}

Im Laufe des nächsten Kapitels, Theoretische Grundlagen, soll Wissen zu diesen
beiden Erweiterungen gesammelt und an praxisbezogenen Beispiel angewandt werden.
Alle weiteren Analyse-Erweiterungen werden in diesem Artikel nur kurz namentlich
erwähnt und erfahren keine weitere Behandlung. Um dies zu gewährleisten sind
einige Theoretische Grundlagen nötig, die zum verstehen der Ergebnisse essenziell
sind. Das Kapitel \ref{chap:hintergund_und_grundlagen} führt demnach in alle
wichtigen Hintergründe ein.