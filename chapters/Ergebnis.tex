\chapter{Ergebnis}
\label{chap:ergebnis} In diesem Kapitel sollen aufbauend auf der Datenbankimplementierung
konkreten Implementierungsbeispiele zu den zwei Analyse-Erweiterungen betrachtet
werden. So lassen sich die Möglichkeiten und Sprachkonstrukte gut aufzeigen. Das
Skript für das Erstellen der Datenbank liegt dem Anhang bei.

\section{Implementierung Window Function}
\label{sec:window_function} Dieses Kapitel beschäftigt sich ausschließlich mit der
konkreten Anwendung der Window-Functions und deren Verwandtheit zu den herkömmlichen
Aggregatfunktionen. Hierzu wird das Beispiel der \textit{LibraryDB} herangezogen.
Zu Beginn sei eine Aggregatfunktion der einfachsten Art gezeigt.

\begin{minipage}{0.55\textwidth}
	\lstset{caption={Aggregatfunktion}, label={list:simpleAgg}} \begin{lstlisting}
SELECT AVG(published_year)
AS avg_year
FROM Books;
	\end{lstlisting}
\end{minipage}
\hfill
\begin{minipage}{0.45\textwidth}
	\centering
	\begin{tabular}{|c|}
		\hline
		\textbf{avg\_year} \\
		\hline
		1947.2500          \\
		\hline
	\end{tabular}
	\captionof{table}{Ergebnis Aggregatfunktion} \label{tab:ergebnis_aggregatfunktion}
\end{minipage}

Diese Query läuft über die Relation \textit{Books} der \textit{LibraryDB}. Dabei
bildet sie den Durchschnitt aller Einträger in der Spalte \textit{published\_year}.
Das Ergebnis dieser Query ist eine einzige Zeile, die den Durchschnitt aller Einträge
bildet. Das Resultat ist in Tabelle \ref{tab:ergebnis_aggregatfunktion} zu sehen.
Im nächsten Schritt kann ein genauerer Einblick gewonnen werden, indem durch die
\texttt{GROUP BY} Klausel die Ergebnismenge in Gruppen unterteilt wird. So ist
erkennbar, welches Genre an Büchern welches durchschnittliche Alter hat. Das
zugehörige Statement sieht wie folgt aus.

\begin{minipage}{0.55\textwidth}
	\lstset{caption={Aggregatfunktion und Group\_By}, label={list:group_by}} \begin{lstlisting}
SELECT AVG(published_year)
AS avg_year
FROM Books
GROUP BY genre;
	\end{lstlisting}
\end{minipage}
\hfill
\begin{minipage}{0.45\textwidth}
	\centering
	\begin{tabular}{|c|c|}
		\hline
		\textbf{genre} & \textbf{avg\_year} \\
		\hline
		Fiction        & 1945.3333          \\
		\hline
		Dystopian      & 1940.5000          \\
		\hline
		Adventure      & 1851.0000          \\
		\hline
		Fantasy        & 1968.5000          \\
		\hline
		Thriller       & 2003.0000          \\
		\hline
	\end{tabular}
	\captionof{table}{Ergebnis Group\_by} \label{tab:ergebnis_group_by}
\end{minipage}

Diese Query liefert genau soviel Records, wie es verschiedenen Einträger in der
Spalte \textit{genre} gibt, die Tabelle \ref{tab:ergebnis_group_by} zeigt einen
Ausschnitt. Es wird dann für jedes Genre ein durchschnittliches Alter der Bücher
angegeben. Für die Nächste Betrachtung soll aus der Aggregatfunktion \texttt{AVG(<..>)}
eine Analyse-Funktion generiert werden. Hierzu ist hinter die Aggregatfunktion
das Schlüsselwort \texttt{OVER()} anzuhängen, wie in \ref{list:empty_over} zu
sehen ist.

\begin{minipage}{0.60\textwidth}
	\lstset{caption={Aggregatfunktion mit Over}, label={list:empty_over}} \begin{lstlisting}
SELECT AVG(published_year) OVER()
AS avg_year
FROM Books
GROUP BY genre;
	\end{lstlisting}
\end{minipage}
\hfill
\begin{minipage}{0.45\textwidth}
	\centering
	\begin{tabular}{|c|}
		\hline
		\textbf{avg\_year} \\
		\hline
		1947.2500          \\
		\hline
		1947.2500          \\
		\hline
		1947.2500          \\
		\hline
		1947.2500          \\
		\hline
	\end{tabular}
	\captionof{table}{Ergebnis Over} \label{tab:ergebnis_over}
\end{minipage}

Nach Betrachtung der Ergebnismenge fällt sofort auf, dass nun ein viel
detaillierter Einblick gewonnen wird. \texttt{OVER()} sorgt dafür, dass für jeden
ursprünglichen Wert der Menge der bestimmte Durchschnittswert angegeben wird.
Anders als bei den klassischen Aggregatfunktionen aus \ref{list:simpleAgg} wird
die Ergebnismenge nicht zusammengefasst. Im Beispiel aus dem Block \ref{list:empty_over}
wird durch das \texttt{Over} bereits ein Fenster über die Ergebnismenge gelegt. In
diesem Fall umschließt das Fenster alle Einträge, da keine genaueren Angaben im
Argument angegeben wurden. Für ein praxisnahes Beispiel mit mehreren Fenstern,
soll eine Ausleihhistorie aufgebaut werden, die angibt welches Mitglied wie viele
Bücher ausgeliehen hat. Hierzu soll die Analyse-Funktion \texttt{ROW\_NUMBER} herangezogen
werden. Das Ziel ist es über die Ergebnismenge der \textit{Borrowed\_Books} Fenster
zu legen und diese anhand der \textit{member\_id} aufzuteilen. \texttt{ROW\_NUMBER}
zählt dann die Zeilen in jedem Fenster. Listing \ref{list:window-function} zeigt
den nötigen Quelltext für dieses Beispiel.

\lstset{caption={Analyse-Funktion mit mehreren Fenstern}, label={list:window-function}}
\begin{lstlisting}
SELECT member_id, book_id, borrow_date, return_date,
ROW_NUMBER() OVER(PARTITION BY member_id) AS borrow_sequence
FROM Borrowed_Books;
\end{lstlisting}
\begin{table}[h]
	\centering
	\begin{tabular}{|c|c|c|c|c|}
		\hline
		\textbf{member\_id} & \textbf{book\_id} & \textbf{borrow\_date} & \textbf{return\_date} & \textbf{borrow\_sequence} \\
		\hline
		1                   & 1                 & 2023-01-01            & 2023-01-15            & 1                         \\
		\hline
		1                   & 4                 & 2023-04-10            & NULL                  & 2                         \\
		\hline
		2                   & 2                 & 2023-02-01            & 2023-02-10            & 1                         \\
		\hline
		2                   & 1                 & 2023-06-15            & NULL                  & 2                         \\
		\hline
		3                   & 3                 & 2023-03-05            & 2023-03-20            & 1                         \\
		\hline
		4                   & 5                 & 2023-05-12            & 2023-06-01            & 1                         \\
		\hline
	\end{tabular}
	\caption{Ergebnis Analyse-Funktion}
	\label{tab:member_borrows}
\end{table}

Durch das Zählen der Zeilen für jedes Fenster entsteht etwas, das in der Fachliteratur
als \textit{running total} beschrieben wird \citep{Nuijten2023}. Die Gruppierung
der Fenster sind anhand der \textit{member\_id} zu erkennen.

Mit der \texttt{ORDER BY}-Klausel kann noch ein Schritt weiter gegangen werden
und die Einträger innerhalb eines Fensters sortiert werden. In dem konkreten Beispiel
der \textit{LibraryDB} soll jedes Fenster nach der \textit{book\_id} sortiert werden.
Die konkrete Syntax für diese Query zeigt der folgenden Bock.

\lstset{caption={Analytische-Funktion und Order By}, label={list:window-function-order-by}}
\begin{lstlisting}
SELECT member_id, book_id, borrow_date, return_date,
ROW_NUMBER() OVER(PARTITION BY member_id ORDER BY book_id)
AS borrow_sequence
FROM Borrowed_Books;
\end{lstlisting}
\begin{table}[h]
	\centering
	\begin{tabular}{|c|c|c|c|c|}
		\hline
		\textbf{member\_id} & \textbf{book\_id} & \textbf{borrow\_date} & \textbf{return\_date} & \textbf{borrow\_sequence} \\
		\hline
		1                   & 1                 & 2023-01-01            & 2023-01-15            & 1                         \\
		\hline
		1                   & 4                 & 2023-04-10            & NULL                  & 2                         \\
		\hline
		2                   & 1                 & 2023-06-15            & NULL                  & 2                         \\
		\hline
		2                   & 2                 & 2023-02-01            & 2023-02-10            & 1                         \\
		\hline
		3                   & 3                 & 2023-03-05            & 2023-03-20            & 1                         \\
		\hline
		4                   & 5                 & 2023-05-12            & 2023-06-01            & 1                         \\
		\hline
	\end{tabular}
	\caption{Ergebnis Order By}
	\label{tab:member_borrows}
\end{table}

Wie zu erkennen ist, wird die \textit{oder\_by\_clause} wie in früheren Kapiteln
bereits beschrieben einfach an die \textit{partition\_by\_clause} Klausel
angehängt. Sie ist somit ein Teil der \textit{analytic\_clause}. Mit Abschluss dieser
Beispielreihe wurde noch lange nicht alle Möglichkeiten der Window-Funktions angesprochen
die Dokumentation von \citet{oracle} beschreibt noch weitere Sprachkonstrukte, die
in dieser Arbeit aber unberücksichtigt belieben.

\section{Implementierung - Rollup and Cube}
\label{sec:implementierung_rollup_and_cube} Die Erweiterungen \texttt{ROLLUP}
und \texttt{CUBE} werden als Erweiterung der \texttt{GROUP BY} Klausel
verstanden. Dieses Kapitel soll nun durch Hilfe der \textit{LibraryDB} die
Möglichkeiten und Sprachkonstrukte näher aufschlüsseln. Dazu sollen die Analogien
zu der herkömmlichen \texttt{GROUP BY} Funktion visualisiert werden. Um mit herkömmlichen
Befehlen Einsicht in mehrere Gruppierungen zu erhalten, so muss für jede
gewünschte Gruppierung ein \texttt{SELECT} geschrieben werden. Der Codeblöcke 
\ref{list:group_date} und \ref{list:group_id} zeigen dies.

\begin{minipage}{0.45\textwidth}
	\lstset{caption={Gruppe Date}, label={list:group_date}} \begin{lstlisting}
SELECT borrow_date,
FROM Borrowed_Books
GROUP BY(borrow_date);
	\end{lstlisting}
\end{minipage}
\hfill
\begin{minipage}{0.45\textwidth}
	\lstset{caption={Gruppe ID}, label={list:group_id}} \begin{lstlisting}
SELECT member_id,
FROM Borrowed_Books
GROUP BY(member_id);
	\end{lstlisting}
\end{minipage}

Um dieses Vorgehen zu vereinfachen, kann auf das Schlüsselwort \texttt{GROUPING
SETS} zurückgegriffen werden, dass die Blöcke \ref{list:group_id} und
\ref{list:group_date} in einer Abfrage vereint. Die Ergebnismenge wird ebenfalls
in einer Relation dargestellt. Alle nicht relevanten Felder werden dann mit
\texttt{NULL} Aufgefüllt.

\begin{lstlisting}[caption={Beispiel eines Grouping sets}, label={list:grouping_sets}]
SELECT borrow_date, member_id,
FROM Borrowed_Books
GROUP BY GROUPING SETS(borrow_date, member_id);
\end{lstlisting}

Mit \texttt{GROUPING SETS} ist es nicht möglich eine Kombination aus den angegebenen
Attributen zu gruppieren. Diese Funktion ist den Erweiterungen \texttt{ROLLUP} und
\texttt{CUBE} überlassen. Um aus einem Set an Gruppierungen einnen \texttt{ROLLUP}
zu erstellen, ist lediglich das Schlüsselwort auszutauschen. Konkret ersetzten
wir also das \texttt{GROUPING SETS} gegen ein \texttt{ROLLUP}. Die Implementierung
kann nun so verändert werden, dass ein praktisches Beispiel sichtbar wird. Mit der
Aggregatfunktion \texttt{COUNT} und der \texttt{ROLLUP} Funktion kann eine Query
gebaut werden, die in einer Ergebnismenge folgenden Daten liefert.

\begin{description}
	\label{bul:aufzählung}
	\item[$\bullet$ 1. Punkt] Wie viele Bücher wurden insgesamt schon ausgeliehen

	\item[$\bullet$ 2. Punkt] Wie viel Bücher wurden an einem bestimmten Tag ausgeliehen

	\item[$\bullet$ 2. Punkt] Wie viel Bücher wurden von einer Person an einem Tag ausgeliehen
\end{description}

\begin{lstlisting}[caption={Beispiel eines ROLLUP}, label={list:rollup}]
SELECT borrow_date, member_id, COUNT(book_id)
	AS total_borrowed
FROM Borrowed_Books
GROUP BY ROLLUP(borrow_date, member_id);
\end{lstlisting}

\begin{table}[h]
	\centering
	\begin{tabular}{|c|c|c|}
		\hline
		\textbf{borrow\_date} & \textbf{member\_id} & \textbf{total\_borrowed} \\
		\hline
		2023-01-01            & 1                   & 2                        \\ % books member 1 on 2023-01-01
		\hline
		2023-01-01            & 2                   & 1                        \\ % books member 2 on 2023-01-01
		\hline
		2023-01-01            & NULL                & 3                        \\ % total books on 2023-01-01
		\hline
		2023-03-05            & 3                   & 1                        \\
		\hline
		2023-03-05            & NULL                & 1                        \\
		\hline
		NULL                  & NULL                & 10                       \\
		\hline
	\end{tabular}
	\caption{Ergebnis Rollup}
	\label{tab:rollup}
\end{table}

Tabelle \ref{tab:rollup} zeigt einen Ausschnitt der Ergebnismenger aus der
Query \ref{list:rollup}. Bei genauerem Betrachten der Ergebnismenge lassen sich die oben
genanten Punkte aus der Aufzählung \ref{bul:aufzählung} wiederfinden. Der erste Punkt (Gesamtverleih)
ist auf die letzte Zeile zurückzuführen, die angibt, dass insgesammt bereits 10 Bücher
verliehen wurden. Das Datum oder das Mitglied wird hier also nicht
berücksichtigt. Der Punkt zwei weist auf alle Einträge hin, die in der Spalte
member\_id ein NULL vorfinden. Beispielsweise wurden am 2023-01-01 insgesamt drei
Bücher verliehen (Zeile drei). Die konkreterste Betrachtung sind dann alle
Spalten, die kein NULL aufweisen. Betrachtet man die erste Zeil, so kann dort herrausgelesen
werden, dass am 2023-01-01 das Mitglied mit der Nummer eins, zwei Bücher ausgeliehen
hat. Ohne die Erweiterunge \texttt{ROLLUP} müsste für jede dieser drei Punkte ein
eigenes SELECT geschrieben werden. Durch die richtige Anordung der Attribute
innerhalb der Rollup Funktion kann also ein immer detaillierterer Einblick
gewonnen werden.

Auch wenn der \texttt{ROLLUP} bereits einen sehr tiefen Einblick mit nur wenigen
Zeilen Code gewährt, gibt es noch ein Konstrukt, dass noch tiefer geht. Durch
ersetzten des Schlüsselwortes \texttt{ROLLUP} gegen \texttt{CUBE} lassen sich
noch mehr Kombinationen der Ergebnismenge anzeigen. Hierzu sei auf das Kapitel ...
verweisen. Damit lässt sich auch ermitteln, wie viele Bücher ein Mitglied unabhängig
vom Tag ausgeliehen hat.

\begin{lstlisting}
SELECT borrow_date, member_id, COUNT(book_id)
	AS total_borrowed
FROM Borrowed_Books
GROUP BY CUBE(borrow_date, member_id);
\end{lstlisting}

\section{Zusätzliche Erweiterungen}
\label{sec:zus_erweiterungen} Weitere Analyse-Erweiterungen, die hier aber unberührt
beilen sollen, sind Beispielsweise die Rekursiven Abfragen. Eine Rekursion ist laut
\citet{benecke1998rekursion} die Definition eines Problems, einer Funktion oder eines
Verhaltens durch sich selbst. Aufbauend auf dem Konzept der \textit{Common Table
Espressions} (CTE) können damit auch in SQL Probleme rekursiv gelöst werden \citep{Ignacio2022}.
Hinzu kommt das Konzept des \textit{Pivot and unpivot}, das \citet{Nuijten2023}
näher beschriebt.