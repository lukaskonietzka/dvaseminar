\chapter{Fazit}
\label{chap:fazit} Durch die hier behandelten Konstrukte bieten sich eine Vielzahl
an Möglichkeiten. Gerade in Bezug auf die Datenanalyse, wie es \citet{FOTACHE2015243}
in ihrer Arbeit \textit{SQL and Data Analysis} beschreiben. Die Erweiterungen
Rollup, Cube und Window-Funktions bieten grundsätzlich eine gute Lesbarkeit des Quelltextes.
Jedoch sind für eine korrekte Implementierung einer Query mit Analyse-Funktion
wesentlich mehr Fachkenntnisse nötig. Es müssen viel mehr einzelne Klauseln verstanden
und betrachtet werden, die alle eine eigene Syntax haben. Ein \texttt{ROLLUP}
oder \texttt{CUBE} ist hier wesentlich einfacher zu handhaben. Diese können
lediglich der \texttt{GROUP BY} Klausel vorangestellt werden. Da die Window-Functions
das deutlich komplexere Sprachkonstrukt bieten, ergeben sich hierdurch
mehr Möglichkeiten und die größere Flexibilität. Die Erweiterungen \texttt{ROLLUP}
und \texttt{CUBE} sind hingegen etwas starrer in ihrer Anwendung. Betrachtet man nur die
Ergebnismenge, so fällt auf, dass das Resultat der komplexeren Window-Functions anfangs leichter
zu interpretieren ist. Bei den Konzepten \texttt{ROLLUP} und \texttt{CUBE} gestaltet sich die
Interpretation der Ergebnisrelation etwas schwieriger. Es muss erst verstanden werden,
wie die \texttt{NULL} Werte zu interpretieren sind. Auch sind die verschiedenen
Eben der Gruppierungen nicht sofort ersichtlich.

Mit Blich auf die Forschungsfrage, lässt sich zusammenfassend feststellen, dass
die Integration dieser Analyse-Erweiterungen einen erheblichen Mehrwert für die
Möglichkeiten in der Datenanalyse bietet, insbesondere in Hinblick darauf, wie
einfach komplexe Abfragen geworden sind. Jedoch wurde auch klar, dass für mehr
Möglichkeiten immer komplexer werdende Sprachkonstrukte nötig sind. Für die Zukunft
wäre eine weitergehende Untersuchung der spezifischen Anwendungsfälle dieser Erweiterungen
sinnvoll, um ihre Potenziale vollständig auszuschöpfen. Auch eine weiterführende
Recherche zu dem Thema rekursive Abfragen würde die Arbeit hervorragend ergänzen.




Wie das Kapitel
\ref{sec:relevant} bereits zu Beginn der Arbeit deutlich gemacht hat, finden die
Analyse-Erweiterungen in der Datananalyse vermehrt Anwendung, da sie einen
analytische Sicht auf die vorliegenden Daten gewähren. Beide hier behandelten Erweiterungen
beiten diese analytische Einsicht, wie an den Beispielen aus Kapitel
\ref{sec:window_function} und \ref{sec:rollup_and_cube} deutlich wird.