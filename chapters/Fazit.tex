\chapter{Fazit}
\label{chap:fazit} Abschließend lässt sich über die hier behandelten Analyse-Erweiterungen
sage, dass beide Formen mächtige Sprachkonstrukte bieten, die weit über die herkömmlichen
SQL Befehle hinaus gehen. Durch diese Konstrukte bieten sich jeodch auch viel
mehr Möglichkeiten. Gerade im Bezug auf die Datananalyse, wie es \citet{FOTACHE2015243}
in ihrer Arbeit \textit{SQL and Data Analysis} beschreiben. Die Erweiterungen Rollup,
Cube und Window Funktions bieten grundsätzlich eine gute Lesbarkeit des
Quelltextes. Jedoch sind für eine korrekte Implementierung einer Query mit Analyse-Funktion
wesentlich mehr Fachkenntnisse nötig wie bei den Konzepten ROLLUP und CUBE. Es
müssen viel mehr einzelne Klauseln verstanden und betrachtet werden, die alle ihre
eigenen Syntax haben. Ein \texttt{ROLLUP} oder \texttt{CUBE} ist hier wesentlich
einfacher zu handhaben. Diese können lediglich der \texttt{GROUP BY} Klausel
vorangestellt werden. Wie das Kapitel \ref{sec:relevant} bereits zu Beginn der Arbeit
deutlich gemacht hat, finden die Analyse-Erweiterungen in der Datananalyse
vermehrt Anwendung, da sie einen analytische Sicht auf die vorliegenden Daten gewähren.
Beide hier behandelten Erweiterungen beiten diese analytische Einsicht, wie an den
Beispielen aus Kapitel \ref{sec:window_function} und \ref{sec:rollup_and_cube} deutlich
wird. Da die Window Funktions das deutlich komplexere Sprachkonstukt bieten, ergeben
sich hierduch auch mehr Möglichkeiten und die größere Flexibilität. Die
Erweiterungen \texttt{ROLLUP} und \texttt{CUBE} sind hingegen etwas starrer in
ihrer Anwendung. Der letze Punkte wirft ein Blick auf die Ergebnismenge, welche
die beiden hier fokussierten Erweiterungen liefern. Die Window funktions sind das
komplexere Sprachkunstrukt, mit vielen Möglichkeiten bezogen auf die Anwendung
und Flexibilität. Die Interpretation der Ergebnismeng ist hierbei umso einfacher.
Die Struktur hat viele ähnlichkeitne zu den weit verbreiteten Aggregatfunktionen
und ist daher eingänig. Bei den Konzepten \texttt{ROLLUP} und \texttt{CUBE} gestaltet
sich die Interpretation der Ergebnisrelation etwas schwieriger. Es muss erst
verstanden werden, wie die \texttt{NULL} Werte zu interpretieren sind. Auch sind
die verschiedenen Eben der Gruppierunge nicht sofort ersichtlich.

Mit Blich auf die Forschungsfrage, lässt sich zusammenfassend feststellen, dass die
Integration dieser Analyse-Erweiterungen einen erheblichen Mehrwert für die Möglichkeiten
in der Datenanalyse bietet, insbesondere in Hinblick darauf, wie einfach komplexe
Abfragen geworden sind. Jedoch wurde auch klar, dass für viele Möglichkeiten immer
komplexer werdende Sprachkonstrukte nötig sind. Für die Zukunft wäre eine
weitergehende Untersuchung der spezifischen Anwendungsfälle dieser Erweiterungen
sinnvoll, um ihre Potenziale vollständig auszuschöpfen. Auch eine weiterführende
Recherche zu dem Thema Rekursive Abfragen würde die Arbiet hervorragend ergänzen.