\chapter{Hintergrund}
Dieses Kapitel soll die Grundsteine legen und in die Konzepte und Theorien
der Analyse-Erweiterungen einführen. Um die Fragestellung dieser Arbeit einordnen zu
können, ist unter anderem  ein Blick auf die Historie hilfreich. Der Hauptgrund für
die Wahl der drei oben genannten Erweiterungen liegt nämlich nicht nur in ihren vielseitigen
Möglichkeiten, sondern auch in ihrer Integration in den SQL-Standard, wie auch von
\cite{grust2017advanced} in seiner Arbeit hervorgehoben wird. Er verdeutlicht auf Seite 10,
dass im Jahre 1987 die erste Version des SQL-Standard veröffentlicht wurde. Mit SQL3 1999
kamen dann die ersten Analyse-Erweiterungen mit den \textit{recursive queries} hinzu.
Wie \cite{melton2001sql} in Kapitel 9.12 ihrer Arbeit beschreiben, wurden die Funktionen
\textit{ROLLUP} und \textit{CUBE} ebenfalls 1999 in den SQL-Standard aufgenommen.
Die \textit{Window Functions} hingegen folgten laut \cite{grust2017advanced} erst
mit der späteren Version im Jahr 2003.









\section{SQL Kurzüberischt}

\section{Window Funktionen}

\subsection{Definition}

\subsection{Aufbau}

\section{Common Table Expressions und rekursive Queries}

\subsection{Common Table Expressions (CTE)}

\subsection{Rekursive Queries}

\section{Rollup and Cube}

\subsection{Grouping Sets}

\subsection{Rollup}

\subsection{Cube}

\section{Dialekte und Unterstüzung}