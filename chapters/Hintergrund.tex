\chapter{Hintergrund und Grundlagen}
\label{chap:hintergund_und_grundlagen} Dieses Kapitel soll die Grundsteine legen
und in die Konzepte und Theorien der Analyse-Erweiterungen einführen. Um die
Fragestellung dieser Arbeit einordnen zu können, ist unter anderem ein Blick auf
die Historie hilfreich. Der Hauptgrund für die Wahl der drei oben genannten
Erweiterungen liegt nämlich nicht nur in ihren vielseitigen Möglichkeiten, sondern
auch in ihrer Integration in den SQL-Standard, wie aus der Arbeit von \citet[S.~10]{grust2017advanced}.
Durch nähere Betrachtung der Arbeit von \cite{grust2017advanced} wird deutlich, dass
im Jahre 1987 die erste Version des SQL-Standards veröffentlicht wurde. Mit SQL3
1999 kamen dann die ersten Analyse-Erweiterungen mit den \textit{recursive
queries} hinzu \citep[S.~10]{grust2017advanced}. Die Erweiterungen \texttt{ROLLUP}
und \texttt{CUBE} wurden ebenfalls im Jahre 1999 in den SQL-Standard aufgenommen
\citep[Kapitel 9.12]{melton2001sql}. \citet[S.~10]{grust2017advanced} verdeutlicht,
dass die \textit{Window Functions} mit einer späteren Version im Jahr 2003 erschienen
sind.

\section{Window Functions}
\label{sec:window_functions} Die \textit{Analytic Functions} oder auch \textit{Window
Functions} genannt, sind das Herzstück der Analyse-Erweiterungen und somit auch
für die Datenanalyse innerhalb von SQL ein wichtiger Bestandteil. Die \textit{Window
Functions} sind so aus eine komplexe Datenanaylse nicht mehr wegzudenken. \citet[Abstract]{cao2012optimization}
bezeichnen sie sind für den aktuellen Stand der Technik als repräsentativ und absolut
notwendig. Hinzu kommt, dass sie laut \citet[Kapitel 8]{kellenberger2019expert} einen
Performancevorteil bietet. Dieser Performancevorteil ergibt sich jedoch laut
\citet[Kapitel 8]{kellenberger2019expert} nicht primär durch die bessere
Laufzeit, sondern vielmehr dadurch, wie einfach komplexe Abfragen damit werden
können.

Für eine prägnante und eindeutige Definition der \textit{Analytic Functions} sei
auf die Dokumentation der \citet{oracle} Datenbank verwiesen, die die \textit{Analytic
Functions} wie folgt beschreiben.
\begin{center}
	\textit{ 'Analytic functions compute an aggreagt value based on a grpoup of
	rows.' } \\ \cite{oracle}

	\textit{ 'The group of rows is called a window and is defined by the analytic-clause.'
	} \\ \cite{oracle}
\end{center}
Wenn man dieser Definition folgt, dann werden darunter also aggregierte Werte
verstanden, die mittels dem Schlüsselwort \texttt{OVER} auf bestimmte Zeilen in
einer Tabelle angewendet wird. So ist es Beispielsweise möglich, auch
benachbarte Datensätze mit einzubeziehen \citep{oracle}.

Die analytischen Funktionen sind stark verwandt mit den weit verbreiteten Aggregatfunktionen
in SQL. Zur Unterscheidung der Aggregatfunktion und den analytischen Funktionen
kann nach \citet{Nuijten2023} die Ergebnismenge betrachtet werden. Sie erläutern,
dass Beispielsweise ein \texttt{COUNT(<...>)} als Aggregatfunktion nur eine
Zeile zurückliefert, während analytische Funktionen die Ergebnismenge nicht verändern
und für jeden Eintrag einen Wert vorsehen \citep{Nuijten2023}.

Die Abarbeitung einer solchen Analyse-Funktion lässt sich in drei Stufen
einteilen, wobei die Sortierung dieser drei Schritte auch die Reihenfolge der Abarbeitung
festlegt (\cite{Nuijten2023}, Building Blocks).
\begin{description}
	\item[$\bullet$ 1. Stufe] auflösen der \texttt{JOIN}-, \texttt{WHERE}-, \texttt{GROUP
		BY}- und \texttt{HAVING}-Klausel \\ (\cite{Nuijten2023}, Building Blocks)

	\item[$\bullet$ 2. Stufe] die analytische Funktion wird auf die Ergebnismenge angewandt
		\\ (\cite{Nuijten2023}, Building Blocks)

	\item[$\bullet$ 3. Stufe] die ORDER BY-Klausel wir auf die Ergebnismenge angewandt
		\\ (\cite{Nuijten2023}, Building Blocks)
\end{description}
Im nächsten Abschnitt werden die Stufen zwei und drei genauer betrachtet und mittels
eines Modells verdeutlicht.

\subsection{Aufbau}
\label{sec:aufbau} Um den Aufbau einer einfachen \textit{Window Function} etwas
genauer verstehen zu können, sei an dieser Stelle nochmals auf die Dokumentation
von \citet{oracle} verwiesen. Hier ist ein konkreter Aufbau einer analytischen Funktion
zu finden, die hier näher betrachtet werden soll.
\begin{figure}[h]
	\centering
	\includegraphics[scale=0.5]{img/aufbauAnalyticFunction.jpg}
	\caption{ Aufbau einer analytischen Funktion | Quelle: \citep{oracle}}
	\label{fig:aufbauAnalyticFunction}
\end{figure}
Der erste Teil beschreibt die konkrete analytische Funktion, die auf die
Ergebnismenge ausgeführt wird. Hier besteht die größte Parallelität zu den klassischen
Aggregatfunktionen. Das Argument einer Analyse-Funktion legt fest, auf welche Spalte
diese angewendet werden soll \citep[S.~110]{schicker2017datenbanken}. Eine grobe
Übersicht über verschiedenen Window-Functions beschreibt \citet[]{ibrahaim23} in
seiner Arbeit. Hierzu unterteilt die möglichen Funktionen übersichtlich in drei verschiedenen
Gruppen. Eine weitere gute Informationsquelle bietet wieder die \citet{oracle} Dokumentation.

Das Schlüsselwort \texttt{OVER} wurde bereits zuvor in diesem Artikel erwähnt. Es
ist der namensgebende Teil der Window-Functions, da so die Fenster über die
Ergebnismenge gelegt werden. Bleibt das Argument der \texttt{OVER()} Funktion
leer, so wird das Fenster über alle Einträge gespannt. Ein Beispiel liefert
dieser Ausschnitt.

\texttt{AVG(<..>) OVER()} \citep{Nuijten2023}

Im Argument der \texttt{OVER} Funktion wird die Ausprägung und die Ordnung des
Fensters definiert. Dieses Argument ist bekannt als \textit{analytic-clause}
\citep{oracle}. Der Aufbau dieser Klausel sei hier beschrieben \citep{oracle}.
\begin{figure}[h]
	\centering
	\includegraphics[scale=0.5]{img/aufbauAnalyticClausel.jpg}
	\caption{ Aufbau einer analytischen Klausel | Quelle: \citep{oracle}}
	\label{fig:aufbauAnalytischeKlausel}
\end{figure}
Aus dem Aufbau der \textit{analytic-clause} \citep{oracle} wird klar, das diese grob
in zwei Teile aufgeteilt wird. Ein \textit{partition-clause} \citep{oracle}, die
die einzelnen Fenster festlegt und die \textit{order-by-clause} \citep{oracle},
welche die Reihenfolge in den einzelnen Fenstern festlegt. Ein etwas generische Beispiel
sieht wie folgt aus.

\texttt{PARTITION BY <..> ORDER BY <..>} \\ \citep[Analytic Functions]{Nuijten2023}

Wenn die gezeigten Aufbauten aus den Abbildungen \ref{fig:aufbauAnalyticFunction}
und \ref{fig:aufbauAnalytischeKlausel} zusammengesetzt werden, entsteht eine
Analyse-Funktion, die bereits eine höhere Ausprägung hat. Für eine korrekte
Fusion der beiden Klauseln wir die analytische Klausel in das Argument der
\texttt{OVER} Funktion eingefügt \citep[Analytic Functions]{Nuijten2023}.

\texttt{AVG(<..>) OVER (Partition BY <..> ORDER BY <..>)} \\ \citep[Analytic
Functions]{Nuijten2023}

Dieses Sprachkonstrukt kann nun in eine bekannte \texttt{SELECT} Klausel
eingebaut werden. Sie wird an der gleichen Stelle eingebaut, an der auch eine
klassische Aggregatfunktion verwendet wird. Für weitere Details sei auf die Dokumentation
von \citet{oracle} verwiesen, die einen tieferen Einblick in den Aufbau der einzelnen
Klauseln gewähren.

% --------------------------------------------------------------------------------------------

\section{Rollup and Cube}
\label{sec:rollup_and_cube} Auch die letzte Analyse-Erweiterung die in diese
Arbeit behandelt werden soll, macht ihrem Namen alle Ehre und gewährt mit wenigen
Zeilen Quelltext, eine viel analytischere Einsicht in die Ergebnismenge, als es mit
herkömmlichen SQL-Befehlen möglich ist \citep[Kapitel 7.2.3]{melton2002advanced}.
\citet[Kapitel 7.2.3]{melton2002advanced} beschreibt diese als mehrdimensionale
Zusammenfassungen und kontrollierte Unterbrechungen von gruppierten Date. Diese Formulierung
lässt erahnen, dass die Konzepte, \texttt{ROLLUP} und \texttt{CUBE} Erweiterungen
der \texttt{GROUP BY} Klausel sind. Bevor mit den eigentlichen Erweiterungen begonnen
werden kann, muss hier ein Grundkonzept vorangestellt werden. Dieses Führt schrittweise
die Schlüsselwörter \texttt{ROLLUP} und \texttt{CUBE} ein. Hierbei handelt es
sich um das Konzept des \texttt{GROUPING SETS}.

\subsection{Grouping Sets}
\label{subsec:grouping_sets} Das Schlüsselwort \texttt{GROUPING SETS} ist
ebenfalls eine Erweiterung der herkömmlichen \texttt{GROUP BY} Klausel wie auch die
\citet{oracle16} Dokumentation belegt. Wie aus dem Namen schon hervorgeht,
handelt es sich nicht um eine einzelne Gruppierung, sondern um ein ganzes Set an
Gruppierungen. Das Set lässt darauf deuten, dass jede Gruppierung nur einmal vorkommt.
Das Schlüsselwort \texttt{GROUPING SETS} erlaubt mehrere Gruppierungen in einer Abfrage.
Jedoch wird nur eine angegebene Gruppe gruppiert und nicht alle möglichen
Gruppen \citet{oracle16}. Für eine Kombination an Gruppierungen sei an dieser
Stelle schon auf die Erweiterungen \texttt{ROLLUP} und \texttt{CUBE} verwiesen.

\texttt{GROUP BY GROUPING SETS (A, B);}

Folgt man der Dokumentation von \citet{oracle16} und dem hier gezeigten Beispiel,
so ergeben sich bei einem \texttt{GROUPING SETS} folgenden Gruppierungen:
\begin{align*}
	\{A\}, \quad \{B\}
\end{align*}
Die Ergebnismenge dieser Gruppierung ist einmal die Gruppierung nach \textit{A}
und einmal nach \textit{B}. Es werden keine Kombinationen gruppiert, nur die einzelnen
Spalten. Zu erwähnen ist hier, dass dies nur eine Möglichkeit von vielen ist,
ein \texttt{GROUPING SETS} zu verwenden \citet{oracle16}.

\subsection{Rollup}
\label{subsec:rollup} Wie aus dem vorherigen Kapitel bereits hervorgeht, ist \texttt{ROLLUP}
eine Erweiterung der \texttt{GROUP BY}- und \texttt{GROUPING SETS}-Klausel.
Einen Rollup ist als Zwischensumme zu betrachten, die von der kleinsten Ebenen bis
hin zur größten "\textit{aufgerollt}" wird, so beschreibt es die \citet{oracle99}.
Die Reihenfolge der Gruppierung folgt der angegebenen Gruppierungsliste. Die
Richtung der Gruppierungsreihenfolge ist von rechts nach links (\cite{oracle16}).

\texttt{GROUP BY ROLLUP (A, B, C);}

Durch genauere Betrachtung des hier gezeigten Beispiels soll die genau Art und
Weise der \texttt{ROLLUP} Gruppierung näher erläutert werden. Hierzu Betrachten
wir die Gruppierungen als Menge.
\begin{align*}
	\{ \}, \{A\}, \{A, B\}, \{A, B, C\}
\end{align*}
Mit einem \texttt{ROLLUP} werden also in diesem Beispiel vier Mengen geliefert.
Die letzte Klammer steht für die leere Menge und damit die Gesamtsumme. Es wird hier
demnach mit der kleinsten Menge begonnen (rechts) und auf die größte aufgerollt (links).

\subsection{Cube}
\label{subsec:cube} Mit einem \texttt{ROLLUP} sind nicht alle möglichen
Kombinationen an Gruppierungen möglich \citep{oracle99}. Die Erweiterung \texttt{ROLLUP}
gibt eine eindeutige Gruppierungsreihenfolge vor, die ohne Ausnahme eingehalten
wird. Wenn eine andere Reihenfolge oder mehrere Gruppierungen gewollt sind, muss
auf das Konzept der \texttt{CUBE} Erweiterung zurückgegriffen werden, das \citet{oracle99}
in ihrer Dokumentation beschreiben. Wird die Methode des \texttt{CUBE} verwendet,
so ist oft von einem Datenwürfel die Reden \citep{oracle99}. Dieser aggregiert
alle möglichen Kombinationen der Gruppierungen, einschließlich der Gesamtsumme und
erstellt so Datenwürfel. Er berücksichtigt nicht die Reihenfolge, der angegebenen
Gruppierungen \citep{oracle99}.

\texttt{GROUP BY CUBE (A, B, C);}

Unter Betrachtung dieses Beispiels ergeben sich aus den Gruppen \textit{A, B, C}
folgende mögliche Kombinationen.
\begin{align*}
	\{ A, B, C\}, \{A, B\}, \{A, C\}, \{B, C\}, \{A\}, \{B\}, \{C\}, \{ \}
\end{align*}
Durch Betrachtung dieser Mengen wird klar, dass die \texttt{CUBE} Erweiterung einen
noch genaueren Einblick in die vorliegenden Datensätze gewährt.
% --------------------------------------------------------------------------------------------