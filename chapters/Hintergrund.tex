\chapter{Hintergrund}
Dieses Kapitel soll die Grundsteine legen und in die Konzepte und Theorien der Analyse-Erweiterungen
einführen. Um die Fragestellung dieser Arbeit einordnen zu können, ist unter anderem
ein Blick auf die Historie hilfreich. Der Hauptgrund für die Wahl der drei oben
genannten Erweiterungen liegt nämlich nicht nur in ihren vielseitigen
Möglichkeiten, sondern auch in ihrer Integration in den SQL-Standard, wie auch
von \cite{grust2017advanced} in seiner Arbeit hervorgehoben wird. Er verdeutlicht
auf Seite 10, dass im Jahre 1987 die erste Version des SQL-Standards
veröffentlicht wurde. Mit SQL3 1999 kamen dann die ersten Analyse-Erweiterungen mit
den \textit{recursive queries} hinzu. Wie \cite{melton2001sql} in Kapitel 9.12 ihrer
Arbeit beschreiben, wurden die Funktionen \textit{ROLLUP} und \textit{CUBE}
ebenfalls 1999 in den SQL-Standard aufgenommen. Die \textit{Window Functions}
hingegen folgten laut \cite{grust2017advanced} erst mit der späteren Version im
Jahr 2003.

\section{SQL Kurzüberischt}
Wird nur ausgeschrieben, wenn wir nicht genug seiten haben

\section{Window Functions}
Die \textit{Analytic Functions} oder auch \textit{Window Functions} genannt,
sind das Herzstück der Analyse-Erweiterungen und somit auch für die Datenanalyse
innerhalb von SQL ein wichtiger Bestandteil. \cite{cao2012optimization} veröffentlichte
eine Arbeit über die Optimierung von \textit{Window Functions} wo er bereits zu
Beginn der Arbeit erwähnt, dass für eine Komplexe Datenanalyse diese SQL Erweiterungen
nicht mehr wegzudenken ist. Er beschreibt in seinem Artikel die \textit{Analytic
Functions} als repräsentativ für den aktuellen Stand der Technik.

Für eine pregnante und eindeutige Definition der \textit{Analytic Functions}
sei auf die Dokumentation der Oracle Datenbank verwiesen, die die \textit{Analytic
Functions} wie folgt beschreiben.

\begin{center}
	\textit{ 'Analytic functions compute an aggreagt value based on a grpoup of
	rows.' } \\ \cite{oracle}

	\textit{ 'The group of rows is called a window and is defined by the analytic-clause.'
	} \\ \cite{oracle}
\end{center}

Wenn man dieser Definition folgt, dann werden darunter also aggregierte Werte
verstanden, die mittels dem Schlüsselwort \textit{OVER} auf bestimmte Zeilen in
einer Tabelle angewendet wird. So ist es Beispielsweise möglich, auch
benachbarte Datensätze mit einzubeziehen.

Die analytischen Funktionen, sind stark verwand mit den weit verbreiteten
Aggregatfunktionen in SQL. Sie bilden in gewisser Weise eine Erweiterung dieser Aggregatfunktionen.
\cite{Nuijten2023} betrachten zur Unterscheidung der Aggregatfunktion und den
analytischen Funktionen die Ergebnismenge. Sie erläutern, dass Beispielsweise
ein \textit{COUNT(*)} als Aggregatfunktion nur eine Zeile zurückliefert, während
analytische Funktionen die Ergebnismenge nicht verändern und für jeden Eintrag
einen Wert angeben.

\cite{Nuijten2023} beschreiben weiter, dass sich die Abarbeitung der Funktionen in
drei Stufen einteilen lässt, wobei dir Sortierung dieser drei Schritte auch die
Reihenfolge der Abarbeitung festlegt.

\begin{description}
	\item[$\bullet$ 1. Stufe] auflösen der JOIN-, WHERE-, GROUP BY- und HAVING-Klausel

	\item[$\bullet$ 2. Stufe] die analytische Funktion wird auf die Ergebnismenge angewnand

	\item[$\bullet$ 3. Stufe] die ORDER BY-Klausel wir auf die Ergebnismenge angewandt
\end{description}

Im nächsten Abschnitt werden die Stufen zwei und drei genauer betrachtet und mittels
eines Modells verdeutlicht.

\subsection{Aufbau}
Um den Aufbau einer einfachen \textit{Window Function} etwas genauer verstehen
zu können, sei an dieser Stelle nochmals auf \cite{oracel} verwiesen.
Hier ist ein konkreter Aufbau einer analytischen Funktion zu finden, die hier näher
betrachtet werden soll.

\begin{figure}[h]
	\centering
	\includegraphics[scale=0.5]{img/aufbauAnalyticFunction.jpg}
	\caption{ Aufbau einer analytischen Funktion | Quelle: \cite{oracle}}
\end{figure}

Der erste Teil beschreibt die konkrete analytische Funktion, die auf die
Erbenismenge ausgeführt wird. Hier besteht die größte Parallelität zu den klassischen
Aggregatfunktionen. Eine grobe Übersicht über verschiedenen \textit{Window
Functions} beschreibt \cite{ibrahaim23} in seiner Arbeit. Hierzu unterteilt er
im Kapitel \textit{Different Types of Window Functions} die möglichen Funktionen
übersichtlich in drei verschiedenen Gruppen,

Das Schlüsselwort \textit{OVER(*)} wurde bereits zuvor in diesem Artikel erwähnt.
Es ist der namensgebende Teil der \textit{Window Functions}, da so die Fenster über
die Ergebnismenge gelegt werden.

Im Argument der \textit{OVER} Funktion definieren wir die Ausprägung des
Fensters und die Ordnung des Fensters. Dieses Argument ist bekannt als \textit{analytic-clause}
und teilt sich laut \cite{oracle} wieder in weitere Teile auf. Der Aufbau dieser
\textit{analytic-clause} sei hier beschrieben, bleibt aber hier unkommentiert.

\begin{figure}[h]
	\centering
	\includegraphics[scale=0.5]{img/aufbauAnalyticClausel.jpg}
	\caption{ Aufbau einer analytischen Klausel | Quelle: \cite{oracle}}
\end{figure}

% ------------------------------------------------------

\section{Common Table Expressions und rekursive Queries}

\subsection{Common Table Expressions (CTE)}

\subsection{Rekursive Queries}

\section{Rollup and Cube}

\subsection{Grouping Sets}

\subsection{Rollup}

\subsection{Cube}

\section{Dialekte und Unterstützung}