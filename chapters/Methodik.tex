\chapter{Methodik}
\label{chap:methodik} Dieses Kapitel soll das Vorgehen und die konkrete Methodik
während der Ausarbeitung dieser Arbeit näher erläutern. Begonnen wurde mit einer
umfassenden Literaturrecherche zum aktuellen Stand der Technik. Anschließend wird
eine Beispielimplementierung eingeführt anhand derer konkrete Beispiele angezeigt
werden.

\section{Literaturrecherche}
\label{sec:literaturrecherche} Die Literaturrecherche bildet das Fundament dieser
Arbeit und liefert alle nötigen Grundlagen und Hintergründe. Es werden die hier
relevanten Konzepte und Methoden zu den Analyse-Erweiterungen eingeführt und
erläutert. Darüber hinaus sind Grundkonzepte der einzelnen Erweiterungen zu finden.
Das Ziel der Recherche war es auch die offiziellen Dokumentationen der großen
DBMS Hersteller mit einzubeziehen.

\section{Implementierung}
\label{sec:implementierung} Bevor mit dem Erstellen einer Beispieldatenbank
begonnen werden kann, muss ein Datenbank-Management-System (DBMS) gewählt werden,
mit dem die Beispiele generiert werden können. Aufgrund der einfachen Installation
und des wenigen Overheads, wird für die folgenden Beispiele eine MYSQL Datenbank
verwendet. Dieses bietet alle Erweiterungen, die in dieser Arbeit näher
betrachtet werden sollen.

Als praxisnahes Beispiel wird auf ein allgemein bekanntes Situation
zurückgegriffen, um die Komplexität zu reduzieren und den Fokus mehr auf die Analyse-Erweiterungen
zu lenken. Im Rahmen dieses Artikels soll die Implementierung einer Datenbank
für eine Bibliothek als Beispiel fungieren. Dieses trägt fortan den Namen \textit{LibraryDB}.
Das Skript für die Erstellung der Datenbank und auch die konkreten Relationen können
dem \hyperref[sec:library_db]{Anhang} entnommen werden. Die Implementierung des
SQL Skripts soll innerhalb dieser Arbeit unberührt bleiben, bilde aber die Grundlage
der Beispiele.