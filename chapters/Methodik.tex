\chapter{Methodik}
\label{chap:methodik} Dieses Kapitel soll das Vorgehen und die konkrete Methodik
während der Ausarbeitung dieser Arbeit näher erläutern. Begonnen wurde mit einer
umfassenden Literaturrecherche zum aktuellen Stand der Technik. Anschließend wird
eine Beispielimplementierung eingeführt anhand derer konkrete Beispiele angezeigt
werden. Zuletzt werden die gezeigten Sprachkonstrukte mittels vordefinierter
Bewertungskriterien eingeordnet.

\section{Literaturrecherche}
\label{sec:literaturrecherche} Die Literaturrecherche bildet das Fundament
dieser Arbeit und liefert alle nötigen Grundlagen und Hintergründe. Es werden die
hier relevanten Konzepte und Methoden zu den Analyse-Erweiterungen eingeführt und
erläutert. Darüber hinaus sind Grundkonzepte der einzelnen Erweiterungen zu
finden. Das Ziel der Recherche war es auch die offiziellen Dokumentationen der großen
DBMS Hersteller mit einzubeziehen.

\section{Implementierung}
\label{sec:implementierung} Bevor mit dem Erstellen einer Beispieldatenbank begonnen
werden kann, muss ein Datenbank-Management-System (DBMS) gewählt werden, mit dem
die Beispiele generiert werden können. Aufgrund der einfachen Installation und
des wenigen Overheads, wird für die folgenden Beispiele eine MYSQL Datenbank verwendet.
Dieses bietet alle Erweiterungen, die in dieser Arbeit näher betrachtet werden sollen.

Als praxisnahes Beispiel wird auf ein allgemein bekanntes Situation zurückgegriffen,
um die Komplexität zu reduzieren und den Fokus mehr auf die Analyse-Erweiterungen
zu lenken. Im Rahmen dieses Artikels soll die Implementierung einer Datenbank für
eine Bibliothek als Beispiel fungieren. Dieses trägt fortan den Namen \textit{LibraryDB}.
Das Skript für die Erstellung der Datenbank und auch die konkreten Relationen
können dem \hyperref[sec:library_db]{Anhang} entnommen werden. Die Implementierung
des SQL Skripts soll innerhalb dieser Arbeit unberührt bleiben, bilde aber die
Grundlage der Beispiele.

\section{Interpretation}
\label{sec:interpretatin} Für eine objektive und einheitliche Bewertung der
Beispiele sollen Parameter festgehalten werden, anhand deren eine Beurteilung möglich
ist. Für Diese Arbeite werden die Kriterien \textit{Lesbarkeit, Flexibilität,
Ergebnismenge} herangezogen. Im ersten Punkt soll als überprüft werden, ob die Sprachkonstrukte
lesbar gestaltet sind. Im zweiten Punkt soll eine Aussage über die Flexibilität
der Queries getroffen werden. Zuletzt soll auch die konkrete Ergebnismenge nicht
zu kurz kommen. Nachdem alle nötigen Rahmenbedingungen gesetzt wurden, kann in den
nächsten beiden Kapiteln auf die konkreten Beispiele eingegangen werden. Die
Auswertung der Beispiele und damit die Bewertung der einzelnen Möglichkeiten und
Sprachkonstrukte sind im Kapitel \ref{chap:fazit}.