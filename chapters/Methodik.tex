\chapter{Methodik}
In dieses Kapitel sollen die fokussierten Analyse-Erweiterungen genauer unter die
Lupe genommen werden. Hierzu werden praxisnahe Beispiele für die Erweiterungen
generiert und herangezogen. Nachdem ein Einblick in die Beispiele gewonnen wurde,
soll anhand von speziellen Bewertungskriterien die Sprachkonstrukte und
Möglichkeiten dieser Analyse-Erweiterungen objektiv bewertet werden. Auch die Ergebnismenge
der verschiedenen Beispiele soll in die Bewertung mit einfließen.

Für solch ein Vorgehen wird im folgenden Kapitel der entsprechende Rahmen gesetzt.

\section{Rahmenbedingungen}
Bevor mit dem Erstellen einer Beispieldatenbank begonnen werden kann, muss ein
Datenbank Management System (DBMS) gewählt werden. Aufgrund der einfachen Installation
wird für die folgenden Beispiele eine MYSQL Datenbank verwendet. Dieses bietet
alle Erweiterungen, die in dieser Arbeit näher betrachtet werden sollen.

Als praxisnahes Beispiel wird auf ein allgemein bekanntes Konzept zurückgegriffen,
um die Komplexität zu reduzieren und den Fokus mehr auf die Analyse-Erweiterungen
zu lenken. Im Rahmen dieses Artikels soll das Konzept einer Bibliothek als Beispiel
fungieren. Dieses trägt fortan den Namen \textit{LibraryDB}. Das logische
Datenmodell teilt sich hier in drei Relationen auf.
\begin{description}
	\item[$\bullet$ 1. Relation] Books

	\item[$\bullet$ 2. Relation] Members

	\item[$\bullet$ 3. Realtion] BorrowedBooks
\end{description}

Die Relation \textit{BorrowedBooks} stellt eine Verbindung zwischen den Tabellen
\textit{Members} und \textit{Books} her. So kann Beispielsweise auch ein
Mitglied mehrere Bücher ausleihen. Die folgende Abbildung zeigt das genaue logische
Datenmodell.

TABELLEN einbauen

Für eine objektive und einheitliche Bewertung der Beispiele sollen zuletzt noch
Parameter festgelegt werden, anhand deren eine Beurteilung möglich ist. Für
Diese Arbeite werden folgende Kriterien herangezogen:
\begin{description}
	\item[$\bullet$ 1. Parameter] Lesbarkeit

	\item[$\bullet$ 2. Parameter] Flexibilität

	\item[$\bullet$ 3. Parameter] Ergebnismenge
\end{description}

Im ersten Punkt soll ein Blick auf die Lesbarkeit der Anfragen geworfen werden. Sind
sie strukturiert und verständlich aufgebaut. Im zweiten Punkt soll eine Aussage
über die Flexibilität der Querys getroffen werden. Zuletzt soll auch die konkrete
Ergebnismenge nicht zu kurz kommen.

Nachdem alle nötigen Rahmenbedingungen gesetzt wurden, kann in den nächsten
beiden Kapiteln auf die konkreten Beispiele eingegangen werden.

\section{Beispiel analytische Funktion}

\section{Beispiel - Rollup and Cube}