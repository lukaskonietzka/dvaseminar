\chapter{Methodik}
In dieses Kapitel sollen die fokusierten Analyse-Erweiterungen genauer unter die
Lupe genommen werden. Hierzu werden praxisnahe Beispiele für die Erweiterungen
genneriert und hernagezogen. Nachdem eine Einblick in die Beispiele gewonnen wurde,
soll anhand von speziellen Bewertungskriterien die Sprachkonstrukte und
Möglichleiten dieser Analyse-Erweiterungen objektiv bewertet werden. Auch die Ergebnismenge
der verschiednen Beispiel soll in die Bewertung mit einfließen.

Für sollch ein Vorgehen, wird im folgenden Kapitel der entsprechende Rahmen
gesetzt.

\section{Rahmenbedingungen}
Befor mit dem erstellen einer Beispieldatenbank begonnen werden kann, muss ein Datenbank
Management System (DBMS) gewählt werden. Aufgrund der einfachen Installation
wird für die folgenden Beispiele eine MYSQL Datenbank verwendet. Dieses bietet alle
Erweiterungen, die in dieser Arbeit näher betrachtet werden sollen.

Als praxisnahes Beispiel wird auf ein allgemein bekanntes konzept
zurückgegriffen um die Komplexität zu reduzieren und den Fokus mehr auf die
Analyseerweiterungen zu lenken. Im Rahmen dieses Artikel soll das Konzept einer Bibliothek
als Beispiel fungieren. Dieses trägt vortan den Namen \textit{LibraryDB}. Das
Datenmodell teilt sich hier in drei Relationen auf.

\section{Beispiel analytische Funktion}

\section{Beispiel - Rekursive Queries}

\section{Beispiel - Rollup and Cube}