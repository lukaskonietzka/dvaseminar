\section{Common Table Expressions und rekursive Queries}
Neben den \textit{Window Functions} sollen hier noch die rekursiven Abfragen,
auch \textit{Recursiv Queries} genannt, behandelt werden. Um eine rekursive Abfrage
schreiben zu können, muss zunächst ein anderes Konzept in SQL verstanden werden.
Die rekursiven Abfragen bauen auf diesem Konzept auf und sind ohne dies nicht einsetzbar.
Die Rede ist von den \textit{Common Table Expressions (CTE)}.

Der erste Abschnitt dieses Kapitels führt demnach in die Theorien und Konzepte der
\textit{CTE} ein und baut darauf auf. Im zweiten Schritt sollen die rekursiven Abfragen
betrachtet werden.

\subsection{Common Table Expressions (CTE)}
Die \textit{Common Table Expressions} sind ein mächtiges Konstrukt in SQL und
vereinfachen komplexe Anfragen. Speziell Abfragen mit einem \textit{subselect}
werden dadurch deutlich einfache.

Für eine aussagekräftige Definition der \textit{CTE} sei auf den Internetartikel
% von \cite{Ignacio2022} verwiesen, der die Frage: "Was ist eine \textit{CTE}" mit
folgendem Satz beantwortet:
\begin{center}
	\textit{{}{}{}{}{}{}{}{}{}{}{}{}{}{}{}{}{}{}{}{}{}{}{}{} 'A Common Table
	Expression (CTE) is like a named subquery. It functions as a virtual table that
	only its main query can access. CTEs can help simplify,
	% shorten, and organize your code.' } \\ \cite{Ignacio2022}
\end{center}
Vereinfacht lässt sich daraus ableite, dass ein CTE ein query mit Namen ist. Es
handelt sich hier also um ein \texttt{SELECT}, das mit einem Namen angesprochen
werden kann. Betrachtet wird es als virtuelle Tabelle, da diese erst zur
Laufzeit des jeweiligen \texttt{SELECT} angelegt wird und nach Abschluss sofort zerstört
wird. Einmal erstell, kann so eine Virtuelle Tabelle wie eine herkömliche Tabelle
behandelt werden. Die
% \textit{main query}, die \cite{Ignacio2022} anspricht, ist das \texttt{SELECT} Statement,
dass direkt nach der CTE Erstellung folgt. Das CTE leitet also eine kleine Abstraktionsschicht
% ein, die hilft, die Vorteil, die aus der Definition von \cite{Ignacio2022}
hervorgehen, einzuhalten.

Ein CTE bedient sich folgender Syntax:

\texttt{WITH cteName (<...>) SELECT mainQuery FROM cteName;}

Für \texttt{cteName} ist der entsprechende Name für die virtuelle Tabelle zu wählen.
Die \texttt{(<...>)} weist darauf hin, dass in der runden Klammer das jeweilige
\texttt{SELECT} Statement steht, welches die Identität der CTE festlegt. \texttt{mainQuery}
ist die Query, die auf die virtuelle Tabelle verweist und unmittelbar nach der CTE
folgt.

\subsection{Rekursive Queries}
Mit dem Wissen aus dem vorherigen Abschnitt soll hier in die rekursiven Abfragen
% eingeführt werden. Eine Rekursion ist laut \cite{benecke1998rekursion} Kapitel zehn
(Rekursion) die Definition eines Problems, einer Funktion oder eines Verhaltens
durch sich selbst. Er schreibt weiter, dass der Begriff Rekursion in der
Informatik oft im Zusammenhang mit Datenstrukturen und Funktionen steht. Dies Trifft
auch auf die CTE. Diese werden Verwendendet um innerhalb einer CTE auf sich
selbst zu referenzieren und so einen rekursiven Aufruf generieren.

\begin{lstlisting} [caption={Rekrusive Abfragen mit CTE | Quelle: \cite{Mohan23}}, label={list:rekursion}]
WITH RECURSIVE cteName -- define CTE
AS (SELECT <...> -- base case
	FROM <...>
	UNION ALL -- combine
	SELECT <...> -- recursive case
	FROM cteName
)
SELECT * FROM cteName; -- main query
\end{lstlisting}

Das Beispiel aus \ref{list:rekursion} soll den grundsätzlichen Aufbau einer rekursiven
% Abfrage zeigen. Wie auch \cite{Mohan23} in seinem Beispiel schon beschrieben hat,
wird zunächst die virtuelle Tabelle (CTE) mit Name definiert. Innerhalb der CTE
gibt es zwei \texttt{SELECT} Statements. Einen \textit{Base Case} und einen \textit{Recursiv
Case}. Die Rekursion entsteht hier im zweiten \texttt{SELECT}, dass auf den CTE referenziert.
Der \texttt{UNION ALL} kombiniert die beiden \texttt{SELECT}. Die \textit{main
query} ist Teil der klassischen CTE und wurde im obrigen Kapitel bereits
behandelt. Es wird hier genutzt um letzten Endes das Ergebnis, dass in der CTE
zusammengebaut wurde, auszugeben.

% --------------------------------------------------------------------------------------------

Rekursion ein fester Bestandteil der Informatik ist. Er betont, dass viele
Algorithmen auf Basis der Rekursion definiert sind (\cite{4460710}, Kapitel 1).
Mit SQL3 (1999) wird die rekursive Query auch in den SQL-Standard eingeführt und
soll hier die dritte Erweiterung sein, die behandelt wird.(\cite{grust2017advanced},
Seite 10)

\begin{description}
	\item[$\bullet$ Recursive Queries] als Erweiterung der Commen Table Extentions(CTE)
\end{description}